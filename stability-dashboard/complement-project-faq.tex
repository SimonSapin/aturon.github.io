\documentclass[]{article}
\usepackage{lmodern}
\usepackage{amssymb,amsmath}
\usepackage{ifxetex,ifluatex}
\usepackage{fixltx2e} % provides \textsubscript
\ifnum 0\ifxetex 1\fi\ifluatex 1\fi=0 % if pdftex
  \usepackage[T1]{fontenc}
  \usepackage[utf8]{inputenc}
\else % if luatex or xelatex
  \ifxetex
    \usepackage{mathspec}
    \usepackage{xltxtra,xunicode}
  \else
    \usepackage{fontspec}
  \fi
  \defaultfontfeatures{Mapping=tex-text,Scale=MatchLowercase}
  \newcommand{\euro}{€}
\fi
% use upquote if available, for straight quotes in verbatim environments
\IfFileExists{upquote.sty}{\usepackage{upquote}}{}
% use microtype if available
\IfFileExists{microtype.sty}{\usepackage{microtype}}{}
\ifxetex
  \usepackage[setpagesize=false, % page size defined by xetex
              unicode=false, % unicode breaks when used with xetex
              xetex]{hyperref}
\else
  \usepackage[unicode=true]{hyperref}
\fi
\hypersetup{breaklinks=true,
            bookmarks=true,
            pdfauthor={},
            pdftitle={Project FAQ},
            colorlinks=true,
            citecolor=blue,
            urlcolor=blue,
            linkcolor=magenta,
            pdfborder={0 0 0}}
\urlstyle{same}  % don't use monospace font for urls
\setlength{\parindent}{0pt}
\setlength{\parskip}{6pt plus 2pt minus 1pt}
\setlength{\emergencystretch}{3em}  % prevent overfull lines
\setcounter{secnumdepth}{5}

\title{Project FAQ}

\begin{document}
\maketitle

0.11.0 (305cf1f1098dc87e346003e6bf7e7cca5705a135 2014-07-10 11:19:20 -0700)

Copyright © 2011-2014 The Rust Project Developers. Licensed under the
\href{http://www.apache.org/licenses/LICENSE-2.0}{Apache License,
Version 2.0} or the \href{http://opensource.org/licenses/MIT}{MIT
license}, at your option.

This file may not be copied, modified, or distributed except according
to those terms.

{
\hypersetup{linkcolor=black}
\setcounter{tocdepth}{3}
\tableofcontents
}
\section{What is this project's goal, in one
sentence?}\label{what-is-this-projects-goal-in-one-sentence}

To design and implement a safe, concurrent, practical, static systems
language.

\section{Why are you doing this?}\label{why-are-you-doing-this}

Existing languages at this level of abstraction and efficiency are
unsatisfactory. In particular:

\begin{itemize}
\itemsep1pt\parskip0pt\parsep0pt
\item
  Too little attention paid to safety.
\item
  Poor concurrency support.
\item
  Lack of practical affordances, too dogmatic about paradigm.
\end{itemize}

\section{What are some non-goals?}\label{what-are-some-non-goals}

\begin{itemize}
\itemsep1pt\parskip0pt\parsep0pt
\item
  To employ any particularly cutting-edge technologies. Old, established
  techniques are better.
\item
  To prize expressiveness, minimalism or elegance above other goals.
  These are desirable but subordinate goals.
\item
  To cover the complete feature-set of C++, or any other language. It
  should provide majority-case features.
\item
  To be 100\% static, 100\% safe, 100\% reflective, or too dogmatic in
  any other sense. Trade-offs exist.
\item
  To run on ``every possible platform''. It must eventually work without
  unnecessary compromises on widely-used hardware and software
  platforms.
\end{itemize}

\section{Is any part of this thing
production-ready?}\label{is-any-part-of-this-thing-production-ready}

No. Feel free to play around, but don't expect completeness or stability
yet. Expect incompleteness and breakage.

\section{Is this a completely Mozilla-planned and orchestrated
thing?}\label{is-this-a-completely-mozilla-planned-and-orchestrated-thing}

No. It started as a Graydon Hoare's part-time side project in 2006 and
remained so for over 3 years. Mozilla got involved in 2009 once the
language was mature enough to run some basic tests and demonstrate the
idea. Though it is sponsored by Mozilla, Rust is developed by a diverse
community of enthusiasts.

\section{What will Mozilla use Rust
for?}\label{what-will-mozilla-use-rust-for}

Mozilla intends to use Rust as a platform for prototyping experimental
browser architectures. Specifically, the hope is to develop a browser
that is more amenable to parallelization than existing ones, while also
being less prone to common C++ coding errors that result in security
exploits. The name of that project is
\emph{\href{http://github.com/mozilla/servo}{Servo}}.

\section{Why a BSD-style permissive license rather than MPL or
tri-license?}\label{why-a-bsd-style-permissive-license-rather-than-mpl-or-tri-license}

\begin{itemize}
\itemsep1pt\parskip0pt\parsep0pt
\item
  Partly due to preference of the original developer (Graydon).
\item
  Partly due to the fact that languages tend to have a wider audience
  and more diverse set of possible embeddings and end-uses than focused,
  coherent products such as web browsers. We'd like to appeal to as many
  of those potential contributors as possible.
\end{itemize}

\section{Why dual MIT/ASL2 license?}\label{why-dual-mitasl2-license}

The Apache license includes important protection against patent
aggression, but it is not compatible with the GPL, version 2. To avoid
problems using Rust with GPL2, it is alternately MIT licensed.

\end{document}
