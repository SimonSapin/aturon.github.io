\documentclass[]{article}
\usepackage{lmodern}
\usepackage{amssymb,amsmath}
\usepackage{ifxetex,ifluatex}
\usepackage{fixltx2e} % provides \textsubscript
\ifnum 0\ifxetex 1\fi\ifluatex 1\fi=0 % if pdftex
  \usepackage[T1]{fontenc}
  \usepackage[utf8]{inputenc}
\else % if luatex or xelatex
  \ifxetex
    \usepackage{mathspec}
    \usepackage{xltxtra,xunicode}
  \else
    \usepackage{fontspec}
  \fi
  \defaultfontfeatures{Mapping=tex-text,Scale=MatchLowercase}
  \newcommand{\euro}{€}
\fi
% use upquote if available, for straight quotes in verbatim environments
\IfFileExists{upquote.sty}{\usepackage{upquote}}{}
% use microtype if available
\IfFileExists{microtype.sty}{\usepackage{microtype}}{}
\ifxetex
  \usepackage[setpagesize=false, % page size defined by xetex
              unicode=false, % unicode breaks when used with xetex
              xetex]{hyperref}
\else
  \usepackage[unicode=true]{hyperref}
\fi
\hypersetup{breaklinks=true,
            bookmarks=true,
            pdfauthor={},
            pdftitle={Language FAQ},
            colorlinks=true,
            citecolor=blue,
            urlcolor=blue,
            linkcolor=magenta,
            pdfborder={0 0 0}}
\urlstyle{same}  % don't use monospace font for urls
\setlength{\parindent}{0pt}
\setlength{\parskip}{6pt plus 2pt minus 1pt}
\setlength{\emergencystretch}{3em}  % prevent overfull lines
\setcounter{secnumdepth}{5}

\title{Language FAQ}

\begin{document}
\maketitle

0.11.0 (305cf1f1098dc87e346003e6bf7e7cca5705a135 2014-07-10 11:19:20 -0700)

Copyright © 2011-2014 The Rust Project Developers. Licensed under the
\href{http://www.apache.org/licenses/LICENSE-2.0}{Apache License,
Version 2.0} or the \href{http://opensource.org/licenses/MIT}{MIT
license}, at your option.

This file may not be copied, modified, or distributed except according
to those terms.

{
\hypersetup{linkcolor=black}
\setcounter{tocdepth}{3}
\tableofcontents
}
\subsection{Are there any big programs written in it yet? I want to read
big
samples.}\label{are-there-any-big-programs-written-in-it-yet-i-want-to-read-big-samples.}

There aren't many large programs yet. The Rust
\href{https://github.com/rust-lang/rust/tree/master/src/librustc}{compiler},
60,000+ lines at the time of writing, is written in Rust. As the oldest
body of Rust code it has gone through many iterations of the language,
and some parts are nicer to look at than others. It may not be the best
code to learn from, but
\href{https://github.com/rust-lang/rust/blob/master/src/librustc/middle/borrowck/}{borrowck}
and
\href{https://github.com/rust-lang/rust/blob/master/src/librustc/middle/resolve.rs}{resolve}
were written recently.

A research browser engine called
\href{https://github.com/mozilla/servo}{Servo}, currently 30,000+ lines
across more than a dozen crates, will be exercising a lot of Rust's
distinctive type-system and concurrency features, and integrating many
native libraries.

Some examples that demonstrate different aspects of the language:

\begin{itemize}
\itemsep1pt\parskip0pt\parsep0pt
\item
  \href{https://github.com/pcwalton/sprocketnes}{sprocketnes}, an NES
  emulator with no GC, using modern Rust conventions
\item
  The language's general-purpose
  \href{https://github.com/rust-lang/rust/blob/master/src/libstd/hash/mod.rs}{hash}
  function, SipHash-2-4. Bit twiddling, OO, macros
\item
  The standard library's
  \href{https://github.com/rust-lang/rust/blob/master/src/libcollections/hashmap.rs}{HashMap},
  a sendable hash map in an OO style
\item
  The extra library's
  \href{https://github.com/rust-lang/rust/blob/master/src/libserialize/json.rs}{json}
  module. Enums and pattern matching
\end{itemize}

You may also be interested in browsing
\href{https://github.com/trending?l=rust}{GitHub's Rust} page.

\subsection{Does it run on Windows?}\label{does-it-run-on-windows}

Yes. All development happens in lock-step on all 3 target platforms.
Using MinGW, not Cygwin. Note that the windows implementation currently
has some limitations: in particular 64-bit build is
\href{https://github.com/rust-lang/rust/issues/1237}{not fully supported
yet}, and all executables created by rustc
\href{https://github.com/rust-lang/rust/issues/11782}{depends on libgcc
DLL at runtime}.

\subsection{Is it OO? How do I do this thing I normally do in an OO
language?}\label{is-it-oo-how-do-i-do-this-thing-i-normally-do-in-an-oo-language}

It is multi-paradigm. Not everything is shoe-horned into a single
abstraction. Many things you can do in OO languages you can do in Rust,
but not everything, and not always using the same abstraction you're
accustomed to.

\subsection{How do you get away with ``no null
pointers''?}\label{how-do-you-get-away-with-no-null-pointers}

Data values in the language can only be constructed through a fixed set
of initializer forms. Each of those forms requires that its inputs
already be initialized. A liveness analysis ensures that local variables
are initialized before use.

\subsection{What is the relationship between a module and a
crate?}\label{what-is-the-relationship-between-a-module-and-a-crate}

\begin{itemize}
\itemsep1pt\parskip0pt\parsep0pt
\item
  A crate is a top-level compilation unit that corresponds to a single
  loadable object.
\item
  A module is a (possibly nested) unit of name-management inside a
  crate.
\item
  A crate contains an implicit, un-named top-level module.
\item
  Recursive definitions can span modules, but not crates.
\item
  Crates do not have global names, only a set of non-unique metadata
  tags.
\item
  There is no global inter-crate namespace; all name management occurs
  within a crate.
\item
  Using another crate binds the root of \emph{its} namespace into the
  user's namespace.
\end{itemize}

\subsection{Why is failure unwinding non-recoverable within a task? Why
not try to ``catch
exceptions''?}\label{why-is-failure-unwinding-non-recoverable-within-a-task-why-not-try-to-catch-exceptions}

In short, because too few guarantees could be made about the dynamic
environment of the catch block, as well as invariants holding in the
unwound heap, to be able to safely resume; we believe that other methods
of signalling and logging errors are more appropriate, with tasks
playing the role of a ``hard'' isolation boundary between separate
heaps.

Rust provides, instead, three predictable and well-defined options for
handling any combination of the three main categories of ``catch''
logic:

\begin{itemize}
\itemsep1pt\parskip0pt\parsep0pt
\item
  Failure \emph{logging} is done by the integrated logging subsystem.
\item
  \emph{Recovery} after a failure is done by trapping a task failure
  from \emph{outside} the task, where other tasks are known to be
  unaffected.
\item
  \emph{Cleanup} of resources is done by RAII-style objects with
  destructors.
\end{itemize}

Cleanup through RAII-style destructors is more likely to work than in
catch blocks anyways, since it will be better tested (part of the
non-error control paths, so executed all the time).

\subsection{Why aren't modules
type-parametric?}\label{why-arent-modules-type-parametric}

We want to maintain the option to parametrize at runtime. We may make
eventually change this limitation, but initially this is how type
parameters were implemented.

\subsection{Why aren't values type-parametric? Why only
items?}\label{why-arent-values-type-parametric-why-only-items}

Doing so would make type inference much more complex, and require the
implementation strategy of runtime parametrization.

\subsection{Why are enumerations nominal and
closed?}\label{why-are-enumerations-nominal-and-closed}

We don't know if there's an obvious, easy, efficient, stock-textbook way
of supporting open or structural disjoint unions. We prefer to stick to
language features that have an obvious and well-explored semantics.

\subsection{Why aren't channels
synchronous?}\label{why-arent-channels-synchronous}

There's a lot of debate on this topic; it's easy to find a proponent of
default-sync or default-async communication, and there are good reasons
for either. Our choice rests on the following arguments:

\begin{itemize}
\itemsep1pt\parskip0pt\parsep0pt
\item
  Part of the point of isolating tasks is to decouple tasks from one
  another, such that assumptions in one task do not cause undue
  constraints (or bugs, if violated!) in another. Temporal coupling is
  as real as any other kind; async-by-default relaxes the default case
  to only \emph{causal} coupling.
\item
  Default-async supports buffering and batching communication, reducing
  the frequency and severity of task-switching and inter-task /
  inter-domain synchronization.
\item
  Default-async with transmittable channels is the lowest-level building
  block on which more-complex synchronization topologies and strategies
  can be built; it is not clear to us that the majority of cases fit the
  2-party full-synchronization pattern rather than some more complex
  multi-party or multi-stage scenario. We did not want to force all
  programs to pay for wiring the former assumption into all
  communications.
\end{itemize}

\subsection{Why are channels half-duplex
(one-way)?}\label{why-are-channels-half-duplex-one-way}

Similar to the reasoning about default-sync: it wires fewer assumptions
into the implementation, that would have to be paid by all use-cases
even if they actually require a more complex communication topology.

\subsection{Why are strings UTF-8 by default? Why not UCS2 or
UCS4?}\label{why-are-strings-utf-8-by-default-why-not-ucs2-or-ucs4}

The \texttt{str} type is UTF-8 because we observe more text in the wild
in this encoding -- particularly in network transmissions, which are
endian-agnostic -- and we think it's best that the default treatment of
I/O not involve having to recode codepoints in each direction.

This does mean that indexed access to a Unicode codepoint inside a
\texttt{str} value is an O(n) operation. On the one hand, this is
clearly undesirable; on the other hand, this problem is full of
trade-offs and we'd like to point a few important qualifications:

\begin{itemize}
\itemsep1pt\parskip0pt\parsep0pt
\item
  Scanning a \texttt{str} for ASCII-range codepoints can still be done
  safely octet-at-a-time, with each indexing operation pulling out a
  \texttt{u8} costing only O(1) and producing a value that can be cast
  and compared to an ASCII-range \texttt{char}. So if you're (say)
  line-breaking on \texttt{'\textbackslash{}n'}, octet-based treatment
  still works. UTF8 was well-designed this way.
\item
  Most ``character oriented'' operations on text only work under very
  restricted language assumptions sets such as ``ASCII-range codepoints
  only''. Outside ASCII-range, you tend to have to use a complex
  (non-constant-time) algorithm for determining linguistic-unit (glyph,
  word, paragraph) boundaries anyways. We recommend using an ``honest''
  linguistically-aware, Unicode-approved algorithm.
\item
  The \texttt{char} type is UCS4. If you honestly need to do a
  codepoint-at-a-time algorithm, it's trivial to write a
  \texttt{type wstr = {[}char{]}}, and unpack a \texttt{str} into it in
  a single pass, then work with the \texttt{wstr}. In other words: the
  fact that the language is not ``decoding to UCS4 by default''
  shouldn't stop you from decoding (or re-encoding any other way) if you
  need to work with that encoding.
\end{itemize}

\subsection{Why are strings, vectors etc. built-in types rather than
(say) special kinds of
trait/impl?}\label{why-are-strings-vectors-etc.-built-in-types-rather-than-say-special-kinds-of-traitimpl}

In each case there is one or more operator, literal constructor,
overloaded use or integration with a built-in control structure that
makes us think it would be awkward to phrase the type in terms of
more-general type constructors. Same as, say, with numbers! But this is
partly an aesthetic call, and we'd be willing to look at a worked-out
proposal for eliminating or rephrasing these special cases.

\subsection{Can Rust code call C code?}\label{can-rust-code-call-c-code}

Yes. Calling C code from Rust is simple and exactly as efficient as
calling C code from C.

\subsection{Can C code call Rust code?}\label{can-c-code-call-rust-code}

Yes. The Rust code has to be exposed via an \texttt{extern} declaration,
which makes it C-ABI compatible. Such a function can be passed to C code
as a function pointer or, if given the \texttt{\#{[}no\_mangle{]}}
attribute to disable symbol mangling, can be called directly from C
code.

\subsection{Why aren't function signatures inferred? Why only local
slots?}\label{why-arent-function-signatures-inferred-why-only-local-slots}

\begin{itemize}
\itemsep1pt\parskip0pt\parsep0pt
\item
  Mechanically, it simplifies the inference algorithm; inference only
  requires looking at one function at a time.
\item
  The same simplification goes double for human readers. A reader does
  not need an IDE running an inference algorithm across an entire crate
  to be able to guess at a function's argument types; it's always
  explicit and nearby.
\end{itemize}

\subsection{Why does a type parameter need explicit trait bounds to
invoke methods on it, when C++ templates do
not?}\label{why-does-a-type-parameter-need-explicit-trait-bounds-to-invoke-methods-on-it-when-c-templates-do-not}

\begin{itemize}
\item
  Requiring explicit bounds means that the compiler can type-check the
  code at the point where the type-parametric item is \emph{defined},
  rather than delaying to when its type parameters are instantiated. You
  know that \emph{any} set of type parameters fulfilling the bounds
  listed in the API will compile. It's an enforced minimal level of
  documentation, and results in very clean error messages.
\item
  Scoping of methods is also a problem. C++ needs
  \href{http://en.wikipedia.org/wiki/Argument-dependent_name_lookup}{Koenig
  (argument dependent) lookup}, which comes with its own host of
  problems. Explicit bounds avoid this issue: traits are explicitly
  imported and then used as bounds on type parameters, so there is a
  clear mapping from the method to its implementation (via the trait and
  the instantiated type).
\item
  Related to the above point: since a parameter explicitly names its
  trait bounds, a single type is able to implement traits whose sets of
  method names overlap, cleanly and unambiguously.
\item
  There is further discussion on
  \href{https://mail.mozilla.org/pipermail/rust-dev/2013-September/005603.html}{this
  thread on the Rust mailing list}.
\end{itemize}

\subsection{Will Rust implement automatic semicolon insertion, like in
Go?}\label{will-rust-implement-automatic-semicolon-insertion-like-in-go}

For simplicity, we do not plan to do so. Implementing automatic
semicolon insertion for Rust would be tricky because the absence of a
trailing semicolon means ``return a value''.

\subsection{How do I get my program to display the output of logging
macros?}\label{how-do-i-get-my-program-to-display-the-output-of-logging-macros}

\textbf{Short answer} set the RUST\_LOG environment variable to the name
of your source file, sans extension.

\begin{verbatim}
rustc hello.rs
export RUST_LOG=hello
./hello
\end{verbatim}

\textbf{Long answer} RUST\_LOG takes a `logging spec' that consists of a
comma-separated list of paths, where a path consists of the crate name
and sequence of module names, each separated by double-colons. For
standalone .rs files the crate is implicitly named after the source
file, so in the above example we were setting RUST\_LOG to the name of
the hello crate. Multiple paths can be combined to control the exact
logging you want to see. For example, when debugging linking in the
compiler you might set
\texttt{RUST\_LOG=rustc::metadata::creader,rustc::util::filesearch,rustc::back::rpath}
For a full description see \href{log/index.html}{the logging crate}.

\end{document}
