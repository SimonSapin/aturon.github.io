\documentclass[]{article}
\usepackage{lmodern}
\usepackage{amssymb,amsmath}
\usepackage{ifxetex,ifluatex}
\usepackage{fixltx2e} % provides \textsubscript
\ifnum 0\ifxetex 1\fi\ifluatex 1\fi=0 % if pdftex
  \usepackage[T1]{fontenc}
  \usepackage[utf8]{inputenc}
\else % if luatex or xelatex
  \ifxetex
    \usepackage{mathspec}
    \usepackage{xltxtra,xunicode}
  \else
    \usepackage{fontspec}
  \fi
  \defaultfontfeatures{Mapping=tex-text,Scale=MatchLowercase}
  \newcommand{\euro}{€}
\fi
% use upquote if available, for straight quotes in verbatim environments
\IfFileExists{upquote.sty}{\usepackage{upquote}}{}
% use microtype if available
\IfFileExists{microtype.sty}{\usepackage{microtype}}{}
\ifxetex
  \usepackage[setpagesize=false, % page size defined by xetex
              unicode=false, % unicode breaks when used with xetex
              xetex]{hyperref}
\else
  \usepackage[unicode=true]{hyperref}
\fi
\hypersetup{breaklinks=true,
            bookmarks=true,
            pdfauthor={},
            pdftitle={The Strings Guide},
            colorlinks=true,
            citecolor=blue,
            urlcolor=blue,
            linkcolor=magenta,
            pdfborder={0 0 0}}
\urlstyle{same}  % don't use monospace font for urls
\setlength{\parindent}{0pt}
\setlength{\parskip}{6pt plus 2pt minus 1pt}
\setlength{\emergencystretch}{3em}  % prevent overfull lines
\setcounter{secnumdepth}{5}

\title{The Strings Guide}

\begin{document}
\maketitle

0.12.0-pre (2c50add48 2014-07-27 23:03:52 -0700)

Copyright © 2011-2014 The Rust Project Developers. Licensed under the
\href{http://www.apache.org/licenses/LICENSE-2.0}{Apache License,
Version 2.0} or the \href{http://opensource.org/licenses/MIT}{MIT
license}, at your option.

This file may not be copied, modified, or distributed except according
to those terms.

{
\hypersetup{linkcolor=black}
\setcounter{tocdepth}{3}
\tableofcontents
}
Strings are an important concept to master in any programming language.
If you come from a managed language background, you may be surprised at
the complexity of string handling in a systems programming language.
Efficient access and allocation of memory for a dynamically sized
structure involves a lot of details. Luckily, Rust has lots of tools to
help us here.

A \textbf{string} is a sequence of unicode scalar values encoded as a
stream of UTF-8 bytes. All strings are guaranteed to be validly-encoded
UTF-8 sequences. Additionally, strings are not null-terminated and can
contain null bytes.

Rust has two main types of strings: \texttt{\&str} and \texttt{String}.

\section{\&str}\label{str}

The first kind is a \texttt{\&str}. This is pronounced a `string slice.'
String literals are of the type \texttt{\&str}:

\texttt{\{rust\} let string = "Hello there.";}

Like any Rust type, string slices have an associated lifetime. A string
literal is a \texttt{\&'static str}. A string slice can be written
without an explicit lifetime in many cases, such as in function
arguments. In these cases the lifetime will be inferred:

\texttt{\{rust\} fn takes\_slice(slice: \&str) \{     println!("Got: \{\}", slice); \}}

Like vector slices, string slices are simply a pointer plus a length.
This means that they're a `view' into an already-allocated string, such
as a \texttt{\&'static str} or a \texttt{String}.

\section{String}\label{string}

A \texttt{String} is a heap-allocated string. This string is growable,
and is also guaranteed to be UTF-8.

```\{rust\} let mut s = ``Hello''.to\_string(); println!(``\{\}'', s);

s.push\_str(``, world.''); println!(``\{\}'', s); ```

You can coerce a \texttt{String} into a \texttt{\&str} with the
\texttt{as\_slice()} method:

```\{rust\} fn takes\_slice(slice: \&str) \{ println!(``Got: \{\}'',
slice); \}

fn main() \{ let s = ``Hello''.to\_string();
takes\_slice(s.as\_slice()); \} ```

You can also get a \texttt{\&str} from a stack-allocated array of bytes:

```\{rust\} use std::str;

let x: \&{[}u8{]} = \&{[}b'a', b'b'{]}; let stack\_str: \&str =
str::from\_utf8(x).unwrap(); ```

\section{Best Practices}\label{best-practices}

\subsection{\texttt{String} vs. \texttt{\&str}}\label{string-vs.-str}

In general, you should prefer \texttt{String} when you need ownership,
and \texttt{\&str} when you just need to borrow a string. This is very
similar to using \texttt{Vec\textless{}T\textgreater{}} vs.
\texttt{\&{[}T{]}}, and \texttt{T} vs \texttt{\&T} in general.

This means starting off with this:

\texttt{\{rust,ignore\} fn foo(s: \&str) \{}

and only moving to this:

\texttt{\{rust,ignore\} fn foo(s: String) \{}

If you have good reason. It's not polite to hold on to ownership you
don't need, and it can make your lifetimes more complex. Furthermore,
you can pass either kind of string into \texttt{foo} by using
\texttt{.as\_slice()} on any \texttt{String} you need to pass in, so the
\texttt{\&str} version is more flexible.

\subsection{Comparisons}\label{comparisons}

To compare a String to a constant string, prefer
\texttt{as\_slice()}\ldots{}

\texttt{\{rust\} fn compare(string: String) \{     if string.as\_slice() == "Hello" \{         println!("yes");     \} \}}

\ldots{} over \texttt{to\_string()}:

\texttt{\{rust\} fn compare(string: String) \{     if string == "Hello".to\_string() \{         println!("yes");     \} \}}

Converting a \texttt{String} to a \texttt{\&str} is cheap, but
converting the \texttt{\&str} to a \texttt{String} involves an
allocation.

\section{Other Documentation}\label{other-documentation}

\begin{itemize}
\itemsep1pt\parskip0pt\parsep0pt
\item
  \href{/std/str/index.html}{the \texttt{\&str} API documentation}
\item
  \href{std/string/index.html}{the \texttt{String} API documentation}
\end{itemize}

\end{document}
