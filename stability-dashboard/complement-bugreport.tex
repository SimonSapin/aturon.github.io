\documentclass[]{article}
\usepackage{lmodern}
\usepackage{amssymb,amsmath}
\usepackage{ifxetex,ifluatex}
\usepackage{fixltx2e} % provides \textsubscript
\ifnum 0\ifxetex 1\fi\ifluatex 1\fi=0 % if pdftex
  \usepackage[T1]{fontenc}
  \usepackage[utf8]{inputenc}
\else % if luatex or xelatex
  \ifxetex
    \usepackage{mathspec}
    \usepackage{xltxtra,xunicode}
  \else
    \usepackage{fontspec}
  \fi
  \defaultfontfeatures{Mapping=tex-text,Scale=MatchLowercase}
  \newcommand{\euro}{€}
\fi
% use upquote if available, for straight quotes in verbatim environments
\IfFileExists{upquote.sty}{\usepackage{upquote}}{}
% use microtype if available
\IfFileExists{microtype.sty}{\usepackage{microtype}}{}
\ifxetex
  \usepackage[setpagesize=false, % page size defined by xetex
              unicode=false, % unicode breaks when used with xetex
              xetex]{hyperref}
\else
  \usepackage[unicode=true]{hyperref}
\fi
\hypersetup{breaklinks=true,
            bookmarks=true,
            pdfauthor={},
            pdftitle={How to submit a Rust bug report},
            colorlinks=true,
            citecolor=blue,
            urlcolor=blue,
            linkcolor=magenta,
            pdfborder={0 0 0}}
\urlstyle{same}  % don't use monospace font for urls
\setlength{\parindent}{0pt}
\setlength{\parskip}{6pt plus 2pt minus 1pt}
\setlength{\emergencystretch}{3em}  % prevent overfull lines
\setcounter{secnumdepth}{5}

\title{How to submit a Rust bug report}

\begin{document}
\maketitle

0.11.0-pre (169c988d09a9d4e46de2b7fead9489e94964c7c7 2014-07-02 18:41:38 +0000)

Copyright © 2011-2014 The Rust Project Developers. Licensed under the
\href{http://www.apache.org/licenses/LICENSE-2.0}{Apache License,
Version 2.0} or the \href{http://opensource.org/licenses/MIT}{MIT
license}, at your option.

This file may not be copied, modified, or distributed except according
to those terms.

{
\hypersetup{linkcolor=black}
\setcounter{tocdepth}{3}
\tableofcontents
}
\section{I think I found a bug in the
compiler!}\label{i-think-i-found-a-bug-in-the-compiler}

If you see this message:
\texttt{error: internal compiler error: unexpected failure}, then you
have definitely found a bug in the compiler. It's also possible that
your code is not well-typed, but if you saw this message, it's still a
bug in error reporting.

If you see a message about an LLVM assertion failure, then you have also
definitely found a bug in the compiler. In both of these cases, it's not
your fault and you should report a bug!

If you see a compiler error message that you think is meant for users to
see, but it confuses you, \emph{that's a bug too}. If it wasn't clear to
you, then it's an error message we want to improve, so please report it
so that we can try to make it better.

\section{How do I know the bug I found isn't a bug that already exists
in the issue
tracker?}\label{how-do-i-know-the-bug-i-found-isnt-a-bug-that-already-exists-in-the-issue-tracker}

If you don't have enough time for a search, then don't worry about that.
Just submit the bug. If it's a duplicate, somebody will notice that and
close it during triage.

If you have the time for it, it would be useful to type the text of the
error message you got
\href{https://github.com/rust-lang/rust/issues}{into the issue tracker
search box} to see if there's an existing bug that resembles your
problem. If there is, and it's an open bug, you can comment on that
issue and say you are also affected. This will encourage the devs to fix
it. But again, don't let this stop you from submitting a bug. We'd
rather have to do the work of closing duplicates than miss out on valid
bug reports.

\section{What information should I include in a bug
report?}\label{what-information-should-i-include-in-a-bug-report}

It generally helps our diagnosis to include your specific OS (for
example: Mac OS X 10.8.3, Windows 7, Ubuntu 12.04) and your hardware
architecture (for example: i686, x86\_64). It's also helpful to
copy/paste the output of re-running the erroneous rustc command with the
\texttt{-v} flag. Finally, if you can run the offending command under
gdb, pasting a stack trace can be useful; to do so, you will need to set
a breakpoint on \texttt{rust\_fail}.

\section{I submitted a bug, but nobody has commented on
it!}\label{i-submitted-a-bug-but-nobody-has-commented-on-it}

This is sad, but does happen sometimes, since we're short-staffed. If
you submit a bug and you haven't received a comment on it within 3
business days, it's entirely reasonable to either ask on the \#rust IRC
channel, or post on the
\href{https://mail.mozilla.org/listinfo/rust-dev}{rust-dev mailing list}
to ask what the status of the bug is.

\end{document}
