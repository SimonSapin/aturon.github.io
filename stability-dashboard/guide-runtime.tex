\documentclass[]{article}
\usepackage{lmodern}
\usepackage{amssymb,amsmath}
\usepackage{ifxetex,ifluatex}
\usepackage{fixltx2e} % provides \textsubscript
\ifnum 0\ifxetex 1\fi\ifluatex 1\fi=0 % if pdftex
  \usepackage[T1]{fontenc}
  \usepackage[utf8]{inputenc}
\else % if luatex or xelatex
  \ifxetex
    \usepackage{mathspec}
    \usepackage{xltxtra,xunicode}
  \else
    \usepackage{fontspec}
  \fi
  \defaultfontfeatures{Mapping=tex-text,Scale=MatchLowercase}
  \newcommand{\euro}{€}
\fi
% use upquote if available, for straight quotes in verbatim environments
\IfFileExists{upquote.sty}{\usepackage{upquote}}{}
% use microtype if available
\IfFileExists{microtype.sty}{\usepackage{microtype}}{}
\usepackage{color}
\usepackage{fancyvrb}
\newcommand{\VerbBar}{|}
\newcommand{\VERB}{\Verb[commandchars=\\\{\}]}
\DefineVerbatimEnvironment{Highlighting}{Verbatim}{commandchars=\\\{\}}
% Add ',fontsize=\small' for more characters per line
\newenvironment{Shaded}{}{}
\newcommand{\KeywordTok}[1]{\textcolor[rgb]{0.00,0.44,0.13}{\textbf{{#1}}}}
\newcommand{\DataTypeTok}[1]{\textcolor[rgb]{0.56,0.13,0.00}{{#1}}}
\newcommand{\DecValTok}[1]{\textcolor[rgb]{0.25,0.63,0.44}{{#1}}}
\newcommand{\BaseNTok}[1]{\textcolor[rgb]{0.25,0.63,0.44}{{#1}}}
\newcommand{\FloatTok}[1]{\textcolor[rgb]{0.25,0.63,0.44}{{#1}}}
\newcommand{\CharTok}[1]{\textcolor[rgb]{0.25,0.44,0.63}{{#1}}}
\newcommand{\StringTok}[1]{\textcolor[rgb]{0.25,0.44,0.63}{{#1}}}
\newcommand{\CommentTok}[1]{\textcolor[rgb]{0.38,0.63,0.69}{\textit{{#1}}}}
\newcommand{\OtherTok}[1]{\textcolor[rgb]{0.00,0.44,0.13}{{#1}}}
\newcommand{\AlertTok}[1]{\textcolor[rgb]{1.00,0.00,0.00}{\textbf{{#1}}}}
\newcommand{\FunctionTok}[1]{\textcolor[rgb]{0.02,0.16,0.49}{{#1}}}
\newcommand{\RegionMarkerTok}[1]{{#1}}
\newcommand{\ErrorTok}[1]{\textcolor[rgb]{1.00,0.00,0.00}{\textbf{{#1}}}}
\newcommand{\NormalTok}[1]{{#1}}
\ifxetex
  \usepackage[setpagesize=false, % page size defined by xetex
              unicode=false, % unicode breaks when used with xetex
              xetex]{hyperref}
\else
  \usepackage[unicode=true]{hyperref}
\fi
\hypersetup{breaklinks=true,
            bookmarks=true,
            pdfauthor={},
            pdftitle={A Guide to the Rust Runtime},
            colorlinks=true,
            citecolor=blue,
            urlcolor=blue,
            linkcolor=magenta,
            pdfborder={0 0 0}}
\urlstyle{same}  % don't use monospace font for urls
\setlength{\parindent}{0pt}
\setlength{\parskip}{6pt plus 2pt minus 1pt}
\setlength{\emergencystretch}{3em}  % prevent overfull lines
\setcounter{secnumdepth}{5}

\title{A Guide to the Rust Runtime}

\begin{document}
\maketitle

0.11.0-pre (169c988d09a9d4e46de2b7fead9489e94964c7c7 2014-07-02 18:41:38 +0000)

Copyright © 2011-2014 The Rust Project Developers. Licensed under the
\href{http://www.apache.org/licenses/LICENSE-2.0}{Apache License,
Version 2.0} or the \href{http://opensource.org/licenses/MIT}{MIT
license}, at your option.

This file may not be copied, modified, or distributed except according
to those terms.

{
\hypersetup{linkcolor=black}
\setcounter{tocdepth}{3}
\tableofcontents
}
Rust includes two runtime libraries in the standard distribution, which
provide a unified interface to primitives such as I/O, but the language
itself does not require a runtime. The compiler is capable of generating
code that works in all environments, even kernel environments. Neither
does the Rust language need a runtime to provide memory safety; the type
system itself is sufficient to write safe code, verified statically at
compile time. The runtime merely uses the safety features of the
language to build a number of convenient and safe high-level
abstractions.

That being said, code without a runtime is often very limited in what it
can do. As a result, Rust's standard libraries supply a set of
functionality that is normally considered the Rust runtime. This guide
will discuss Rust's user-space runtime, how to use it, and what it can
do.

\section{What is the runtime?}\label{what-is-the-runtime}

The Rust runtime can be viewed as a collection of code which enables
services like I/O, task spawning, TLS, etc. It's essentially an
ephemeral collection of objects which enable programs to perform common
tasks more easily. The actual implementation of the runtime itself is
mostly a sparse set of opt-in primitives that are all self-contained and
avoid leaking their abstractions into libraries.

The current runtime is the engine behind these features (not a
comprehensive list):

\begin{itemize}
\itemsep1pt\parskip0pt\parsep0pt
\item
  I/O
\item
  Task spawning
\item
  Message passing
\item
  Task synchronization
\item
  Task-local storage
\item
  Logging
\item
  Local heaps (GC heaps)
\item
  Task unwinding
\end{itemize}

\subsection{What is the runtime
accomplishing?}\label{what-is-the-runtime-accomplishing}

The runtime is designed with a few goals in mind:

\begin{itemize}
\item
  Rust libraries should work in a number of environments without having
  to worry about the exact details of the environment itself. Two
  commonly referred to environments are the M:N and 1:1 environments.
  Since the Rust runtime was first designed, it has supported M:N
  threading, and it has since gained 1:1 support as well.
\item
  The runtime should not enforce separate ``modes of compilation'' in
  order to work in multiple circumstances. It is an explicit goal that
  you compile a Rust library once and use it forever (in all
  environments).
\item
  The runtime should be fast. There should be no architectural design
  barrier which is preventing programs from running at optimal speeds.
  It is not a goal for the runtime to be written ``as fast as can be''
  at every moment in time. For example, no claims will be made that the
  current implementation of the runtime is the fastest it will ever be.
  This goal is simply to prevent any architectural roadblock from
  hindering performance.
\item
  The runtime should be nearly invisible. The design of the runtime
  should not encourage direct interaction with it, and using the runtime
  should be essentially transparent to libraries. This does not mean it
  should be impossible to query the runtime, but rather it should be
  unconventional.
\end{itemize}

\section{Architecture of the runtime}\label{architecture-of-the-runtime}

This section explains the current architecture of the Rust runtime. It
has evolved over the development of Rust through many iterations, and
this is simply the documentation of the current iteration.

\subsection{A local task}\label{a-local-task}

The core abstraction of the Rust runtime is the task. A task represents
a ``thread'' of execution of Rust code, but it does not necessarily
correspond to an OS thread. Most runtime services are accessed through
the local task, allowing for runtime policy decisions to be made on a
per-task basis.

A consequence of this decision is to require all Rust code using the
standard library to have a local \texttt{Task} structure available to
them. This \texttt{Task} is stored in the OS's thread local storage (OS
TLS) to allow for efficient access to it.

It has also been decided that the presence or non-presence of a local
\texttt{Task} is essentially the \emph{only} assumption that the runtime
can make. Almost all runtime services are routed through this local
structure.

This requirement of a local task is a core assumption on behalf of
\emph{all} code using the standard library, hence it is defined in the
standard library itself.

\subsection{I/O}\label{io}

When dealing with I/O in general, there are a few flavors by which it
can be dealt with, and not all flavors are right for all situations. I/O
is also a tricky topic that is nearly impossible to get consistent
across all environments. As a result, a Rust task is not guaranteed to
have access to I/O, and it is not even guaranteed what the
implementation of the I/O will be.

This conclusion implies that I/O \emph{cannot} be defined in the
standard library. The standard library does, however, provide the
interface to I/O that all Rust tasks are able to consume.

This interface is implemented differently for various flavors of tasks,
and is designed with a focus around synchronous I/O calls. This
architecture does not fundamentally prevent other forms of I/O from
being defined, but it is not done at this time.

The I/O interface that the runtime must provide can be found in the
\href{std/rt/rtio/trait.IoFactory.html}{std::rt::rtio} module. Note that
this interface is \emph{unstable}, and likely always will be.

\subsection{Task Spawning}\label{task-spawning}

A frequent operation performed by tasks is to spawn a child task to
perform some work. This is the means by which parallelism is enabled in
Rust. This decision of how to spawn a task is not a general decision,
and is hence a local decision to the task (not defined in the standard
library).

Task spawning is interpreted as ``spawning a sibling'' and is enabled
through the high level interface in \texttt{std::task}. The child task
can be configured accordingly, and runtime implementations must respect
these options when spawning a new task.

Another local task operation is dealing with the runnable state of the
task itself. This frequently comes up when the question is ``how do I
block a task?'' or ``how do I wake up a task?''. These decisions are
inherently local to the task itself, yet again implying that they are
not defined in the standard library.

\subsection{The \texttt{Runtime} trait and the \texttt{Task}
structure}\label{the-runtime-trait-and-the-task-structure}

The full complement of runtime features is defined by the
\href{std/rt/trait.Runtime.html}{\texttt{Runtime} trait} and the
\href{std/rt/task/struct.Task.html}{\texttt{Task} struct}. A
\texttt{Task} is constant among all runtime implementations, but each
runtime implements has its own implementation of the \texttt{Runtime}
trait.

The local \texttt{Task} stores the runtime value inside of itself, and
then ownership dances ensue to invoke methods on the runtime.

\section{Implementations of the
runtime}\label{implementations-of-the-runtime}

The Rust distribution provides two implementations of the runtime. These
two implementations are generally known as 1:1 threading and M:N
threading.

As with many problems in computer science, there is no right answer in
this question of which implementation of the runtime to choose. Each
implementation has its benefits and each has its drawbacks. The
descriptions below are meant to inform programmers about what the
implementation provides and what it doesn't provide in order to make an
informed decision about which to choose.

\subsection{1:1 - using \texttt{libnative}}\label{using-libnative}

The library \texttt{libnative} is an implementation of the runtime built
upon native OS threads plus libc blocking I/O calls. This is called 1:1
threading because each user-space thread corresponds to exactly one
kernel thread.

In this model, each Rust task corresponds to one OS thread, and each I/O
object essentially corresponds to a file descriptor (or the equivalent
of the platform you're running on).

Some benefits to using libnative are:

\begin{itemize}
\itemsep1pt\parskip0pt\parsep0pt
\item
  Guaranteed interop with FFI bindings. If a C library you are using
  blocks the thread to do I/O (such as a database driver), then this
  will not interfere with other Rust tasks (because only the OS thread
  will be blocked).
\item
  Less I/O overhead as opposed to M:N in some cases. Not all M:N I/O is
  guaranteed to be ``as fast as can be'', and some things (like
  filesystem APIs) are not truly asynchronous on all platforms, meaning
  that the M:N implementation may incur more overhead than a 1:1
  implementation.
\end{itemize}

\subsection{M:N - using \texttt{libgreen}}\label{mn---using-libgreen}

The library \texttt{libgreen} implements the runtime with ``green
threads'' on top of the asynchronous I/O framework
\href{https://github.com/joyent/libuv/}{libuv}. The M in M:N threading
is the number of OS threads that a process has, and the N is the number
of Rust tasks. In this model, N Rust tasks are multiplexed among M OS
threads, and context switching is implemented in user-space.

The primary concern of an M:N runtime is that a Rust task cannot block
itself in a syscall. If this happens, then the entire OS thread is
frozen and unavailable for running more Rust tasks, making this a
(M-1):N runtime (and you can see how this can reach 0/deadlock). By
using asynchronous I/O under the hood (all I/O still looks synchronous
in terms of code), OS threads are never blocked until the appropriate
time comes.

Upon reading \texttt{libgreen}, you may notice that there is no I/O
implementation inside of the library, but rather just the infrastructure
for maintaining a set of green schedulers which switch among Rust tasks.
The actual I/O implementation is found in \texttt{librustuv} which are
the Rust bindings to libuv. This distinction is made to allow for other
I/O implementations not built on libuv (but none exist at this time).

Some benefits of using libgreen are:

\begin{itemize}
\itemsep1pt\parskip0pt\parsep0pt
\item
  Fast task spawning. When using M:N threading, spawning a new task can
  avoid executing a syscall entirely, which can lead to more efficient
  task spawning times.
\item
  Fast task switching. Because context switching is implemented in
  user-space, all task contention operations (mutexes, channels, etc)
  never execute syscalls, leading to much faster implementations and
  runtimes. An efficient context switch also leads to higher throughput
  servers than 1:1 threading because tasks can be switched out much more
  efficiently.
\end{itemize}

\subsubsection{Pools of Schedulers}\label{pools-of-schedulers}

M:N threading is built upon the concept of a pool of M OS threads (which
libgreen refers to as schedulers), able to run N Rust tasks. This
abstraction is encompassed in libgreen's
\href{green/struct.SchedPool.html}{\texttt{SchedPool}} type. This type
allows for fine-grained control over the pool of schedulers which will
be used to run Rust tasks.

In addition the \texttt{SchedPool} type is the \emph{only} way through
which a new M:N task can be spawned. Sibling tasks to Rust tasks
themselves (created through \texttt{std::task::spawn}) will be spawned
into the same pool of schedulers that the original task was home to. New
tasks must previously have some form of handle into the pool of
schedulers in order to spawn a new task.

\subsection{Which to choose?}\label{which-to-choose}

With two implementations of the runtime available, a choice obviously
needs to be made to see which will be used. The compiler itself will
always by-default link to one of these runtimes.

Having a default decision made in the compiler is done out of necessity
and convenience. The compiler's decision of runtime to link to is
\emph{not} an endorsement of one over the other. As always, this
decision can be overridden.

For example, this program will be linked to ``the default runtime''. The
current default runtime is to use libnative.

\begin{Shaded}
\begin{Highlighting}[]
\KeywordTok{fn} \NormalTok{main() \{\}}
\end{Highlighting}
\end{Shaded}

\subsubsection{Force booting with
libgreen}\label{force-booting-with-libgreen}

In this example, the \texttt{main} function will be booted with I/O
support powered by libuv. This is done by linking to the \texttt{rustuv}
crate and specifying the \texttt{rustuv::event\_loop} function as the
event loop factory.

To create a pool of green tasks which have no I/O support, you may shed
the \texttt{rustuv} dependency and use the
\texttt{green::basic::event\_loop} function instead of
\texttt{rustuv::event\_loop}. All tasks will have no I/O support, but
they will still be able to deschedule/reschedule (use channels, locks,
etc).

\begin{Shaded}
\begin{Highlighting}[]
\KeywordTok{extern} \NormalTok{crate green;}
\KeywordTok{extern} \NormalTok{crate rustuv;}

\OtherTok{#[}\NormalTok{start}\OtherTok{]}
\KeywordTok{fn} \NormalTok{start(argc: }\KeywordTok{int}\NormalTok{, argv: *const *const }\KeywordTok{u8}\NormalTok{) -> }\KeywordTok{int} \NormalTok{\{}
    \NormalTok{green::start(argc, argv, rustuv::event_loop, main)}
\NormalTok{\}}

\KeywordTok{fn} \NormalTok{main() \{\}}
\end{Highlighting}
\end{Shaded}

\subsubsection{Force booting with
libnative}\label{force-booting-with-libnative}

This program's \texttt{main} function will always be booted with
libnative, running inside of an OS thread.

\begin{Shaded}
\begin{Highlighting}[]
\KeywordTok{extern} \NormalTok{crate native;}

\OtherTok{#[}\NormalTok{start}\OtherTok{]}
\KeywordTok{fn} \NormalTok{start(argc: }\KeywordTok{int}\NormalTok{, argv: *const *const }\KeywordTok{u8}\NormalTok{) -> }\KeywordTok{int} \NormalTok{\{}
    \NormalTok{native::start(argc, argv, main)}
\NormalTok{\}}

\KeywordTok{fn} \NormalTok{main() \{\}}
\end{Highlighting}
\end{Shaded}

\section{Finding the runtime}\label{finding-the-runtime}

The actual code for the runtime is spread out among a few locations:

\begin{itemize}
\itemsep1pt\parskip0pt\parsep0pt
\item
  \href{std/rt/index.html}{std::rt}
\item
  \href{native/index.html}{libnative}
\item
  \href{green/index.html}{libgreen}
\item
  \href{rustuv/index.html}{librustuv}
\end{itemize}

\end{document}
