\documentclass[]{article}
\usepackage{lmodern}
\usepackage{amssymb,amsmath}
\usepackage{ifxetex,ifluatex}
\usepackage{fixltx2e} % provides \textsubscript
\ifnum 0\ifxetex 1\fi\ifluatex 1\fi=0 % if pdftex
  \usepackage[T1]{fontenc}
  \usepackage[utf8]{inputenc}
\else % if luatex or xelatex
  \ifxetex
    \usepackage{mathspec}
    \usepackage{xltxtra,xunicode}
  \else
    \usepackage{fontspec}
  \fi
  \defaultfontfeatures{Mapping=tex-text,Scale=MatchLowercase}
  \newcommand{\euro}{€}
\fi
% use upquote if available, for straight quotes in verbatim environments
\IfFileExists{upquote.sty}{\usepackage{upquote}}{}
% use microtype if available
\IfFileExists{microtype.sty}{\usepackage{microtype}}{}
\ifxetex
  \usepackage[setpagesize=false, % page size defined by xetex
              unicode=false, % unicode breaks when used with xetex
              xetex]{hyperref}
\else
  \usepackage[unicode=true]{hyperref}
\fi
\hypersetup{breaklinks=true,
            bookmarks=true,
            pdfauthor={},
            pdftitle={The Rust Macros Guide},
            colorlinks=true,
            citecolor=blue,
            urlcolor=blue,
            linkcolor=magenta,
            pdfborder={0 0 0}}
\urlstyle{same}  % don't use monospace font for urls
\setlength{\parindent}{0pt}
\setlength{\parskip}{6pt plus 2pt minus 1pt}
\setlength{\emergencystretch}{3em}  % prevent overfull lines
\setcounter{secnumdepth}{5}

\title{The Rust Macros Guide}

\begin{document}
\maketitle

0.12.0-pre (2c50add48 2014-07-27 23:03:52 -0700)

Copyright © 2011-2014 The Rust Project Developers. Licensed under the
\href{http://www.apache.org/licenses/LICENSE-2.0}{Apache License,
Version 2.0} or the \href{http://opensource.org/licenses/MIT}{MIT
license}, at your option.

This file may not be copied, modified, or distributed except according
to those terms.

{
\hypersetup{linkcolor=black}
\setcounter{tocdepth}{3}
\tableofcontents
}
\section{Introduction}\label{introduction}

Functions are the primary tool that programmers can use to build
abstractions. Sometimes, however, programmers want to abstract over
compile-time syntax rather than run-time values. Macros provide
syntactic abstraction. For an example of how this can be useful,
consider the following two code fragments, which both pattern-match on
their input and both return early in one case, doing nothing otherwise:

\begin{verbatim}
# enum T { SpecialA(uint), SpecialB(uint) }
# fn f() -> uint {
# let input_1 = SpecialA(0);
# let input_2 = SpecialA(0);
match input_1 {
    SpecialA(x) => { return x; }
    _ => {}
}
// ...
match input_2 {
    SpecialB(x) => { return x; }
    _ => {}
}
# return 0u;
# }
\end{verbatim}

This code could become tiresome if repeated many times. However, no
function can capture its functionality to make it possible to abstract
the repetition away. Rust's macro system, however, can eliminate the
repetition. Macros are lightweight custom syntax extensions, themselves
defined using the \texttt{macro\_rules!} syntax extension. The following
\texttt{early\_return} macro captures the pattern in the above code:

\begin{verbatim}
# #![feature(macro_rules)]
# enum T { SpecialA(uint), SpecialB(uint) }
# fn f() -> uint {
# let input_1 = SpecialA(0);
# let input_2 = SpecialA(0);
macro_rules! early_return(
    ($inp:expr $sp:ident) => ( // invoke it like `(input_5 SpecialE)`
        match $inp {
            $sp(x) => { return x; }
            _ => {}
        }
    );
)
// ...
early_return!(input_1 SpecialA);
// ...
early_return!(input_2 SpecialB);
# return 0;
# }
# fn main() {}
\end{verbatim}

Macros are defined in pattern-matching style: in the above example, the
text \texttt{(\$inp:expr \$sp:ident)} that appears on the left-hand side
of the \texttt{=\textgreater{}} is the \emph{macro invocation syntax}, a
pattern denoting how to write a call to the macro. The text on the
right-hand side of the \texttt{=\textgreater{}}, beginning with
\texttt{match \$inp}, is the \emph{macro transcription syntax}: what the
macro expands to.

\section{Invocation syntax}\label{invocation-syntax}

The macro invocation syntax specifies the syntax for the arguments to
the macro. It appears on the left-hand side of the
\texttt{=\textgreater{}} in a macro definition. It conforms to the
following rules:

\begin{enumerate}
\def\labelenumi{\arabic{enumi}.}
\itemsep1pt\parskip0pt\parsep0pt
\item
  It must be surrounded by parentheses.
\item
  \texttt{\$} has special meaning (described below).
\item
  The \texttt{()}s, \texttt{{[}{]}}s, and \texttt{\{\}}s it contains
  must balance. For example, \texttt{({[})} is forbidden.
\end{enumerate}

Otherwise, the invocation syntax is free-form.

To take as an argument a fragment of Rust code, write \texttt{\$}
followed by a name (for use on the right-hand side), followed by a
\texttt{:}, followed by a \emph{fragment specifier}. The fragment
specifier denotes the sort of fragment to match. The most common
fragment specifiers are:

\begin{itemize}
\itemsep1pt\parskip0pt\parsep0pt
\item
  \texttt{ident} (an identifier, referring to a variable or item.
  Examples: \texttt{f}, \texttt{x}, \texttt{foo}.)
\item
  \texttt{expr} (an expression. Examples: \texttt{2 + 2};
  \texttt{if true then \{ 1 \} else \{ 2 \}}; \texttt{f(42)}.)
\item
  \texttt{ty} (a type. Examples: \texttt{int},
  \texttt{Vec\textless{}(char, String)\textgreater{}}, \texttt{\&T}.)
\item
  \texttt{pat} (a pattern, usually appearing in a \texttt{match} or on
  the left-hand side of a declaration. Examples: \texttt{Some(t)};
  \texttt{(17, 'a')}; \texttt{\_}.)
\item
  \texttt{block} (a sequence of actions. Example:
  \texttt{\{ log(error, "hi"); return 12; \}})
\end{itemize}

The parser interprets any token that's not preceded by a \texttt{\$}
literally. Rust's usual rules of tokenization apply,

So \texttt{(\$x:ident -\textgreater{} ((\$e:expr)))}, though excessively
fancy, would designate a macro that could be invoked like:
\texttt{my\_macro!(i-\textgreater{}(( 2+2 )))}.

\subsection{Invocation location}\label{invocation-location}

A macro invocation may take the place of (and therefore expand to) an
expression, an item, or a statement. The Rust parser will parse the
macro invocation as a ``placeholder'' for whichever of those three
nonterminals is appropriate for the location.

At expansion time, the output of the macro will be parsed as whichever
of the three nonterminals it stands in for. This means that a single
macro might, for example, expand to an item or an expression, depending
on its arguments (and cause a syntax error if it is called with the
wrong argument for its location). Although this behavior sounds
excessively dynamic, it is known to be useful under some circumstances.

\section{Transcription syntax}\label{transcription-syntax}

The right-hand side of the \texttt{=\textgreater{}} follows the same
rules as the left-hand side, except that a \texttt{\$} need only be
followed by the name of the syntactic fragment to transcribe into the
macro expansion; its type need not be repeated.

The right-hand side must be enclosed by delimiters, which the
transcriber ignores. Therefore \texttt{() =\textgreater{} ((1,2,3))} is
a macro that expands to a tuple expression,
\texttt{() =\textgreater{} (let \$x=\$val)} is a macro that expands to a
statement, and \texttt{() =\textgreater{} (1,2,3)} is a macro that
expands to a syntax error (since the transcriber interprets the
parentheses on the right-hand-size as delimiters, and \texttt{1,2,3} is
not a valid Rust expression on its own).

Except for permissibility of \texttt{\$name} (and \texttt{\$(...)*},
discussed below), the right-hand side of a macro definition is ordinary
Rust syntax. In particular, macro invocations (including invocations of
the macro currently being defined) are permitted in expression,
statement, and item locations. However, nothing else about the code is
examined or executed by the macro system; execution still has to wait
until run-time.

\subsection{Interpolation location}\label{interpolation-location}

The interpolation \texttt{\$argument\_name} may appear in any location
consistent with its fragment specifier (i.e., if it is specified as
\texttt{ident}, it may be used anywhere an identifier is permitted).

\section{Multiplicity}\label{multiplicity}

\subsection{Invocation}\label{invocation}

Going back to the motivating example, recall that \texttt{early\_return}
expanded into a \texttt{match} that would \texttt{return} if the
\texttt{match}'s scrutinee matched the ``special case'' identifier
provided as the second argument to \texttt{early\_return}, and do
nothing otherwise. Now suppose that we wanted to write a version of
\texttt{early\_return} that could handle a variable number of
``special'' cases.

The syntax \texttt{\$(...)*} on the left-hand side of the
\texttt{=\textgreater{}} in a macro definition accepts zero or more
occurrences of its contents. It works much like the \texttt{*} operator
in regular expressions. It also supports a separator token (a
comma-separated list could be written \texttt{\$(...),*}), and
\texttt{+} instead of \texttt{*} to mean ``at least one''.

\begin{verbatim}
# #![feature(macro_rules)]
# enum T { SpecialA(uint),SpecialB(uint),SpecialC(uint),SpecialD(uint)}
# fn f() -> uint {
# let input_1 = SpecialA(0);
# let input_2 = SpecialA(0);
macro_rules! early_return(
    ($inp:expr, [ $($sp:ident)|+ ]) => (
        match $inp {
            $(
                $sp(x) => { return x; }
            )+
            _ => {}
        }
    );
)
// ...
early_return!(input_1, [SpecialA|SpecialC|SpecialD]);
// ...
early_return!(input_2, [SpecialB]);
# return 0;
# }
# fn main() {}
\end{verbatim}

\subsubsection{Transcription}\label{transcription}

As the above example demonstrates, \texttt{\$(...)*} is also valid on
the right-hand side of a macro definition. The behavior of \texttt{*} in
transcription, especially in cases where multiple \texttt{*}s are
nested, and multiple different names are involved, can seem somewhat
magical and intuitive at first. The system that interprets them is
called ``Macro By Example''. The two rules to keep in mind are (1) the
behavior of \texttt{\$(...)*} is to walk through one ``layer'' of
repetitions for all of the \texttt{\$name}s it contains in lockstep, and
(2) each \texttt{\$name} must be under at least as many
\texttt{\$(...)*}s as it was matched against. If it is under more, it'll
be repeated, as appropriate.

\subsection{Parsing limitations}\label{parsing-limitations}

For technical reasons, there are two limitations to the treatment of
syntax fragments by the macro parser:

\begin{enumerate}
\def\labelenumi{\arabic{enumi}.}
\itemsep1pt\parskip0pt\parsep0pt
\item
  The parser will always parse as much as possible of a Rust syntactic
  fragment. For example, if the comma were omitted from the syntax of
  \texttt{early\_return!} above, \texttt{input\_1 {[}} would've been
  interpreted as the beginning of an array index. In fact, invoking the
  macro would have been impossible.
\item
  The parser must have eliminated all ambiguity by the time it reaches a
  \texttt{\$name:fragment\_specifier} declaration. This limitation can
  result in parse errors when declarations occur at the beginning of, or
  immediately after, a \texttt{\$(...)*}. For example, the grammar
  \texttt{\$(\$t:ty)* \$e:expr} will always fail to parse because the
  parser would be forced to choose between parsing \texttt{t} and
  parsing \texttt{e}. Changing the invocation syntax to require a
  distinctive token in front can solve the problem. In the above
  example, \texttt{\$(T \$t:ty)* E \$e:exp} solves the problem.
\end{enumerate}

\section{Macro argument pattern
matching}\label{macro-argument-pattern-matching}

\subsection{Motivation}\label{motivation}

Now consider code like the following:

\begin{verbatim}
# #![feature(macro_rules)]
# enum T1 { Good1(T2, uint), Bad1}
# struct T2 { body: T3 }
# enum T3 { Good2(uint), Bad2}
# fn f(x: T1) -> uint {
match x {
    Good1(g1, val) => {
        match g1.body {
            Good2(result) => {
                // complicated stuff goes here
                return result + val;
            },
            _ => fail!("Didn't get good_2")
        }
    }
    _ => return 0 // default value
}
# }
# fn main() {}
\end{verbatim}

All the complicated stuff is deeply indented, and the error-handling
code is separated from matches that fail. We'd like to write a macro
that performs a match, but with a syntax that suits the problem better.
The following macro can solve the problem:

\begin{verbatim}
# #![feature(macro_rules)]
macro_rules! biased_match (
    // special case: `let (x) = ...` is illegal, so use `let x = ...` instead
    ( ($e:expr) ~ ($p:pat) else $err:stmt ;
      binds $bind_res:ident
    ) => (
        let $bind_res = match $e {
            $p => ( $bind_res ),
            _ => { $err }
        };
    );
    // more than one name; use a tuple
    ( ($e:expr) ~ ($p:pat) else $err:stmt ;
      binds $( $bind_res:ident ),*
    ) => (
        let ( $( $bind_res ),* ) = match $e {
            $p => ( $( $bind_res ),* ),
            _ => { $err }
        };
    )
)

# enum T1 { Good1(T2, uint), Bad1}
# struct T2 { body: T3 }
# enum T3 { Good2(uint), Bad2}
# fn f(x: T1) -> uint {
biased_match!((x)       ~ (Good1(g1, val)) else { return 0 };
              binds g1, val )
biased_match!((g1.body) ~ (Good2(result) )
                  else { fail!("Didn't get good_2") };
              binds result )
// complicated stuff goes here
return result + val;
# }
# fn main() {}
\end{verbatim}

This solves the indentation problem. But if we have a lot of chained
matches like this, we might prefer to write a single macro invocation.
The input pattern we want is clear:

\begin{verbatim}
# #![feature(macro_rules)]
# fn main() {}
# macro_rules! b(
    ( $( ($e:expr) ~ ($p:pat) else $err:stmt ; )*
      binds $( $bind_res:ident ),*
    )
# => (0))
\end{verbatim}

However, it's not possible to directly expand to nested match
statements. But there is a solution.

\subsection{The recursive approach to macro
writing}\label{the-recursive-approach-to-macro-writing}

A macro may accept multiple different input grammars. The first one to
successfully match the actual argument to a macro invocation is the one
that ``wins''.

In the case of the example above, we want to write a recursive macro to
process the semicolon-terminated lines, one-by-one. So, we want the
following input patterns:

\begin{verbatim}
# #![feature(macro_rules)]
# macro_rules! b(
    ( binds $( $bind_res:ident ),* )
# => (0))
# fn main() {}
\end{verbatim}

\ldots{}and:

\begin{verbatim}
# #![feature(macro_rules)]
# fn main() {}
# macro_rules! b(
    (    ($e     :expr) ~ ($p     :pat) else $err     :stmt ;
      $( ($e_rest:expr) ~ ($p_rest:pat) else $err_rest:stmt ; )*
      binds  $( $bind_res:ident ),*
    )
# => (0))
\end{verbatim}

The resulting macro looks like this. Note that the separation into
\texttt{biased\_match!} and \texttt{biased\_match\_rec!} occurs only
because we have an outer piece of syntax (the \texttt{let}) which we
only want to transcribe once.

\begin{verbatim}
# #![feature(macro_rules)]
# fn main() {

macro_rules! biased_match_rec (
    // Handle the first layer
    (   ($e     :expr) ~ ($p     :pat) else $err     :stmt ;
     $( ($e_rest:expr) ~ ($p_rest:pat) else $err_rest:stmt ; )*
     binds $( $bind_res:ident ),*
    ) => (
        match $e {
            $p => {
                // Recursively handle the next layer
                biased_match_rec!($( ($e_rest) ~ ($p_rest) else $err_rest ; )*
                                  binds $( $bind_res ),*
                )
            }
            _ => { $err }
        }
    );
    // Produce the requested values
    ( binds $( $bind_res:ident ),* ) => ( ($( $bind_res ),*) )
)

// Wrap the whole thing in a `let`.
macro_rules! biased_match (
    // special case: `let (x) = ...` is illegal, so use `let x = ...` instead
    ( $( ($e:expr) ~ ($p:pat) else $err:stmt ; )*
      binds $bind_res:ident
    ) => (
        let $bind_res = biased_match_rec!(
            $( ($e) ~ ($p) else $err ; )*
            binds $bind_res
        );
    );
    // more than one name: use a tuple
    ( $( ($e:expr) ~ ($p:pat) else $err:stmt ; )*
      binds  $( $bind_res:ident ),*
    ) => (
        let ( $( $bind_res ),* ) = biased_match_rec!(
            $( ($e) ~ ($p) else $err ; )*
            binds $( $bind_res ),*
        );
    )
)


# enum T1 { Good1(T2, uint), Bad1}
# struct T2 { body: T3 }
# enum T3 { Good2(uint), Bad2}
# fn f(x: T1) -> uint {
biased_match!(
    (x)       ~ (Good1(g1, val)) else { return 0 };
    (g1.body) ~ (Good2(result) ) else { fail!("Didn't get Good2") };
    binds val, result )
// complicated stuff goes here
return result + val;
# }
# }
\end{verbatim}

This technique applies to many cases where transcribing a result all at
once is not possible. The resulting code resembles ordinary functional
programming in some respects, but has some important differences from
functional programming.

The first difference is important, but also easy to forget: the
transcription (right-hand) side of a \texttt{macro\_rules!} rule is
literal syntax, which can only be executed at run-time. If a piece of
transcription syntax does not itself appear inside another macro
invocation, it will become part of the final program. If it is inside a
macro invocation (for example, the recursive invocation of
\texttt{biased\_match\_rec!}), it does have the opportunity to affect
transcription, but only through the process of attempted pattern
matching.

The second, related, difference is that the evaluation order of macros
feels ``backwards'' compared to ordinary programming. Given an
invocation \texttt{m1!(m2!())}, the expander first expands \texttt{m1!},
giving it as input the literal syntax \texttt{m2!()}. If it transcribes
its argument unchanged into an appropriate position (in particular, not
as an argument to yet another macro invocation), the expander will then
proceed to evaluate \texttt{m2!()} (along with any other macro
invocations \texttt{m1!(m2!())} produced).

\section{Hygiene}\label{hygiene}

To prevent clashes, rust implements
\href{http://en.wikipedia.org/wiki/Hygienic_macro}{hygienic macros}.

As an example, \texttt{loop} and \texttt{for-loop} labels (discussed in
the lifetimes guide) will not clash. The following code will print
``Hello!'' only once:

\begin{verbatim}
#![feature(macro_rules)]

macro_rules! loop_x (
    ($e: expr) => (
        // $e will not interact with this 'x
        'x: loop {
            println!("Hello!");
            $e
        }
    );
)

fn main() {
    'x: loop {
        loop_x!(break 'x);
        println!("I am never printed.");
    }
}
\end{verbatim}

The two \texttt{'x} names did not clash, which would have caused the
loop to print ``I am never printed'' and to run forever.

\section{A final note}\label{a-final-note}

Macros, as currently implemented, are not for the faint of heart. Even
ordinary syntax errors can be more difficult to debug when they occur
inside a macro, and errors caused by parse problems in generated code
can be very tricky. Invoking the \texttt{log\_syntax!} macro can help
elucidate intermediate states, invoking \texttt{trace\_macros!(true)}
will automatically print those intermediate states out, and passing the
flag \texttt{-\/-pretty expanded} as a command-line argument to the
compiler will show the result of expansion.

\end{document}
