\documentclass[]{article}
\usepackage{lmodern}
\usepackage{amssymb,amsmath}
\usepackage{ifxetex,ifluatex}
\usepackage{fixltx2e} % provides \textsubscript
\ifnum 0\ifxetex 1\fi\ifluatex 1\fi=0 % if pdftex
  \usepackage[T1]{fontenc}
  \usepackage[utf8]{inputenc}
\else % if luatex or xelatex
  \ifxetex
    \usepackage{mathspec}
    \usepackage{xltxtra,xunicode}
  \else
    \usepackage{fontspec}
  \fi
  \defaultfontfeatures{Mapping=tex-text,Scale=MatchLowercase}
  \newcommand{\euro}{€}
\fi
% use upquote if available, for straight quotes in verbatim environments
\IfFileExists{upquote.sty}{\usepackage{upquote}}{}
% use microtype if available
\IfFileExists{microtype.sty}{\usepackage{microtype}}{}
\ifxetex
  \usepackage[setpagesize=false, % page size defined by xetex
              unicode=false, % unicode breaks when used with xetex
              xetex]{hyperref}
\else
  \usepackage[unicode=true]{hyperref}
\fi
\hypersetup{breaklinks=true,
            bookmarks=true,
            pdfauthor={},
            pdftitle={The Rust Tasks and Communication Guide},
            colorlinks=true,
            citecolor=blue,
            urlcolor=blue,
            linkcolor=magenta,
            pdfborder={0 0 0}}
\urlstyle{same}  % don't use monospace font for urls
\setlength{\parindent}{0pt}
\setlength{\parskip}{6pt plus 2pt minus 1pt}
\setlength{\emergencystretch}{3em}  % prevent overfull lines
\setcounter{secnumdepth}{5}

\title{The Rust Tasks and Communication Guide}

\begin{document}
\maketitle

0.11.0 (305cf1f1098dc87e346003e6bf7e7cca5705a135 2014-07-10 11:19:20 -0700)

Copyright © 2011-2014 The Rust Project Developers. Licensed under the
\href{http://www.apache.org/licenses/LICENSE-2.0}{Apache License,
Version 2.0} or the \href{http://opensource.org/licenses/MIT}{MIT
license}, at your option.

This file may not be copied, modified, or distributed except according
to those terms.

{
\hypersetup{linkcolor=black}
\setcounter{tocdepth}{3}
\tableofcontents
}
\section{Introduction}\label{introduction}

Rust provides safe concurrency through a combination of lightweight,
memory-isolated tasks and message passing. This guide will describe the
concurrency model in Rust, how it relates to the Rust type system, and
introduce the fundamental library abstractions for constructing
concurrent programs.

Rust tasks are not the same as traditional threads: rather, they are
considered \emph{green threads}, lightweight units of execution that the
Rust runtime schedules cooperatively onto a small number of operating
system threads. On a multi-core system Rust tasks will be scheduled in
parallel by default. Because tasks are significantly cheaper to create
than traditional threads, Rust can create hundreds of thousands of
concurrent tasks on a typical 32-bit system. In general, all Rust code
executes inside a task, including the \texttt{main} function.

In order to make efficient use of memory Rust tasks have dynamically
sized stacks. A task begins its life with a small amount of stack space
(currently in the low thousands of bytes, depending on platform), and
acquires more stack as needed. Unlike in languages such as C, a Rust
task cannot accidentally write to memory beyond the end of the stack,
causing crashes or worse.

Tasks provide failure isolation and recovery. When a fatal error occurs
in Rust code as a result of an explicit call to \texttt{fail!()}, an
assertion failure, or another invalid operation, the runtime system
destroys the entire task. Unlike in languages such as Java and C++,
there is no way to \texttt{catch} an exception. Instead, tasks may
monitor each other for failure.

Tasks use Rust's type system to provide strong memory safety guarantees.
In particular, the type system guarantees that tasks cannot share
mutable state with each other. Tasks communicate with each other by
transferring \emph{owned} data through the global \emph{exchange heap}.

\subsection{A note about the
libraries}\label{a-note-about-the-libraries}

While Rust's type system provides the building blocks needed for safe
and efficient tasks, all of the task functionality itself is implemented
in the standard and sync libraries, which are still under development
and do not always present a consistent or complete interface.

For your reference, these are the standard modules involved in Rust
concurrency at this writing:

\begin{itemize}
\itemsep1pt\parskip0pt\parsep0pt
\item
  \href{std/task/index.html}{\texttt{std::task}} - All code relating to
  tasks and task scheduling,
\item
  \href{std/comm/index.html}{\texttt{std::comm}} - The message passing
  interface,
\item
  \href{sync/struct.DuplexStream.html}{\texttt{sync::DuplexStream}} - An
  extension of \texttt{pipes::stream} that allows both sending and
  receiving,
\item
  \href{sync/struct.Arc.html}{\texttt{sync::Arc}} - The Arc (atomically
  reference counted) type, for safely sharing immutable data,
\item
  \href{sync/raw/struct.Semaphore.html}{\texttt{sync::Semaphore}} - A
  counting, blocking, bounded-waiting semaphore,
\item
  \href{sync/mutex/index.html}{\texttt{sync::Mutex}} - A blocking,
  bounded-waiting, mutual exclusion lock with an associated FIFO
  condition variable,
\item
  \href{sync/struct.RWLock.html}{\texttt{sync::RWLock}} - A blocking,
  no-starvation, reader-writer lock with an associated condvar,
\item
  \href{sync/struct.Barrier.html}{\texttt{sync::Barrier}} - A barrier
  enables multiple tasks to synchronize the beginning of some
  computation,
\item
  \href{sync/struct.TaskPool.html}{\texttt{sync::TaskPool}} - A task
  pool abstraction,
\item
  \href{sync/struct.Future.html}{\texttt{sync::Future}} - A type
  encapsulating the result of a computation which may not be complete,
\item
  \href{sync/one/index.html}{\texttt{sync::one}} - A ``once
  initialization'' primitive
\item
  \href{sync/mutex/index.html}{\texttt{sync::mutex}} - A proper mutex
  implementation regardless of the ``flavor of task'' which is acquiring
  the lock.
\end{itemize}

\section{Basics}\label{basics}

The programming interface for creating and managing tasks lives in the
\texttt{task} module of the \texttt{std} library, and is thus available
to all Rust code by default. At its simplest, creating a task is a
matter of calling the \texttt{spawn} function with a closure argument.
\texttt{spawn} executes the closure in the new task.

\begin{verbatim}
# use std::task::spawn;

// Print something profound in a different task using a named function
fn print_message() { println!("I am running in a different task!"); }
spawn(print_message);

// Print something profound in a different task using a `proc` expression
// The `proc` expression evaluates to an (unnamed) owned closure.
// That closure will call `println!(...)` when the spawned task runs.

spawn(proc() println!("I am also running in a different task!") );
\end{verbatim}

In Rust, there is nothing special about creating tasks: a task is not a
concept that appears in the language semantics. Instead, Rust's type
system provides all the tools necessary to implement safe concurrency:
particularly, \emph{owned types}. The language leaves the implementation
details to the standard library.

The \texttt{spawn} function has a very simple type signature:
\texttt{fn spawn(f: proc())}. Because it accepts only owned closures,
and owned closures contain only owned data, \texttt{spawn} can safely
move the entire closure and all its associated state into an entirely
different task for execution. Like any closure, the function passed to
\texttt{spawn} may capture an environment that it carries across tasks.

\begin{verbatim}
# use std::task::spawn;
# fn generate_task_number() -> int { 0 }
// Generate some state locally
let child_task_number = generate_task_number();

spawn(proc() {
    // Capture it in the remote task
    println!("I am child number {}", child_task_number);
});
\end{verbatim}

\subsection{Communication}\label{communication}

Now that we have spawned a new task, it would be nice if we could
communicate with it. Recall that Rust does not have shared mutable
state, so one task may not manipulate variables owned by another task.
Instead we use \emph{pipes}.

A pipe is simply a pair of endpoints: one for sending messages and
another for receiving messages. Pipes are low-level communication
building-blocks and so come in a variety of forms, each one appropriate
for a different use case. In what follows, we cover the most commonly
used varieties.

The simplest way to create a pipe is to use the \texttt{channel}
function to create a \texttt{(Sender, Receiver)} pair. In Rust parlance,
a \emph{sender} is a sending endpoint of a pipe, and a \emph{receiver}
is the receiving endpoint. Consider the following example of calculating
two results concurrently:

\begin{verbatim}
# use std::task::spawn;

let (tx, rx): (Sender<int>, Receiver<int>) = channel();

spawn(proc() {
    let result = some_expensive_computation();
    tx.send(result);
});

some_other_expensive_computation();
let result = rx.recv();
# fn some_expensive_computation() -> int { 42 }
# fn some_other_expensive_computation() {}
\end{verbatim}

Let's examine this example in detail. First, the \texttt{let} statement
creates a stream for sending and receiving integers (the left-hand side
of the \texttt{let}, \texttt{(tx, rx)}, is an example of a
\emph{destructuring let}: the pattern separates a tuple into its
component parts).

\begin{verbatim}
let (tx, rx): (Sender<int>, Receiver<int>) = channel();
\end{verbatim}

The child task will use the sender to send data to the parent task,
which will wait to receive the data on the receiver. The next statement
spawns the child task.

\begin{verbatim}
# use std::task::spawn;
# fn some_expensive_computation() -> int { 42 }
# let (tx, rx) = channel();
spawn(proc() {
    let result = some_expensive_computation();
    tx.send(result);
});
\end{verbatim}

Notice that the creation of the task closure transfers \texttt{tx} to
the child task implicitly: the closure captures \texttt{tx} in its
environment. Both \texttt{Sender} and \texttt{Receiver} are sendable
types and may be captured into tasks or otherwise transferred between
them. In the example, the child task runs an expensive computation, then
sends the result over the captured channel.

Finally, the parent continues with some other expensive computation,
then waits for the child's result to arrive on the receiver:

\begin{verbatim}
# fn some_other_expensive_computation() {}
# let (tx, rx) = channel::<int>();
# tx.send(0);
some_other_expensive_computation();
let result = rx.recv();
\end{verbatim}

The \texttt{Sender} and \texttt{Receiver} pair created by
\texttt{channel} enables efficient communication between a single sender
and a single receiver, but multiple senders cannot use a single
\texttt{Sender} value, and multiple receivers cannot use a single
\texttt{Receiver} value. What if our example needed to compute multiple
results across a number of tasks? The following program is ill-typed:

\begin{verbatim}
# fn some_expensive_computation() -> int { 42 }
let (tx, rx) = channel();

spawn(proc() {
    tx.send(some_expensive_computation());
});

// ERROR! The previous spawn statement already owns the sender,
// so the compiler will not allow it to be captured again
spawn(proc() {
    tx.send(some_expensive_computation());
});
\end{verbatim}

Instead we can clone the \texttt{tx}, which allows for multiple senders.

\begin{verbatim}
let (tx, rx) = channel();

for init_val in range(0u, 3) {
    // Create a new channel handle to distribute to the child task
    let child_tx = tx.clone();
    spawn(proc() {
        child_tx.send(some_expensive_computation(init_val));
    });
}

let result = rx.recv() + rx.recv() + rx.recv();
# fn some_expensive_computation(_i: uint) -> int { 42 }
\end{verbatim}

Cloning a \texttt{Sender} produces a new handle to the same channel,
allowing multiple tasks to send data to a single receiver. It upgrades
the channel internally in order to allow this functionality, which means
that channels that are not cloned can avoid the overhead required to
handle multiple senders. But this fact has no bearing on the channel's
usage: the upgrade is transparent.

Note that the above cloning example is somewhat contrived since you
could also simply use three \texttt{Sender} pairs, but it serves to
illustrate the point. For reference, written with multiple streams, it
might look like the example below.

\begin{verbatim}
# use std::task::spawn;

// Create a vector of ports, one for each child task
let rxs = Vec::from_fn(3, |init_val| {
    let (tx, rx) = channel();
    spawn(proc() {
        tx.send(some_expensive_computation(init_val));
    });
    rx
});

// Wait on each port, accumulating the results
let result = rxs.iter().fold(0, |accum, rx| accum + rx.recv() );
# fn some_expensive_computation(_i: uint) -> int { 42 }
\end{verbatim}

\subsection{Backgrounding computations:
Futures}\label{backgrounding-computations-futures}

With \texttt{sync::Future}, rust has a mechanism for requesting a
computation and getting the result later.

The basic example below illustrates this.

\begin{verbatim}
use std::sync::Future;

# fn main() {
# fn make_a_sandwich() {};
fn fib(n: u64) -> u64 {
    // lengthy computation returning an uint
    12586269025
}

let mut delayed_fib = Future::spawn(proc() fib(50));
make_a_sandwich();
println!("fib(50) = {}", delayed_fib.get())
# }
\end{verbatim}

The call to \texttt{future::spawn} returns immediately a \texttt{future}
object regardless of how long it takes to run \texttt{fib(50)}. You can
then make yourself a sandwich while the computation of \texttt{fib} is
running. The result of the execution of the method is obtained by
calling \texttt{get} on the future. This call will block until the value
is available (\emph{i.e.} the computation is complete). Note that the
future needs to be mutable so that it can save the result for next time
\texttt{get} is called.

Here is another example showing how futures allow you to background
computations. The workload will be distributed on the available cores.

\begin{verbatim}
# use std::sync::Future;
fn partial_sum(start: uint) -> f64 {
    let mut local_sum = 0f64;
    for num in range(start*100000, (start+1)*100000) {
        local_sum += (num as f64 + 1.0).powf(-2.0);
    }
    local_sum
}

fn main() {
    let mut futures = Vec::from_fn(1000, |ind| Future::spawn( proc() { partial_sum(ind) }));

    let mut final_res = 0f64;
    for ft in futures.mut_iter()  {
        final_res += ft.get();
    }
    println!("π^2/6 is not far from : {}", final_res);
}
\end{verbatim}

\subsection{Sharing immutable data without copy:
Arc}\label{sharing-immutable-data-without-copy-arc}

To share immutable data between tasks, a first approach would be to only
use pipes as we have seen previously. A copy of the data to share would
then be made for each task. In some cases, this would add up to a
significant amount of wasted memory and would require copying the same
data more than necessary.

To tackle this issue, one can use an Atomically Reference Counted
wrapper (\texttt{Arc}) as implemented in the \texttt{sync} library of
Rust. With an Arc, the data will no longer be copied for each task. The
Arc acts as a reference to the shared data and only this reference is
shared and cloned.

Here is a small example showing how to use Arcs. We wish to run
concurrently several computations on a single large vector of floats.
Each task needs the full vector to perform its duty.

\begin{verbatim}
use std::rand;
use std::sync::Arc;

fn pnorm(nums: &[f64], p: uint) -> f64 {
    nums.iter().fold(0.0, |a, b| a + b.powf(p as f64)).powf(1.0 / (p as f64))
}

fn main() {
    let numbers = Vec::from_fn(1000000, |_| rand::random::<f64>());
    let numbers_arc = Arc::new(numbers);

    for num in range(1u, 10) {
        let task_numbers = numbers_arc.clone();

        spawn(proc() {
            println!("{}-norm = {}", num, pnorm(task_numbers.as_slice(), num));
        });
    }
}
\end{verbatim}

The function \texttt{pnorm} performs a simple computation on the vector
(it computes the sum of its items at the power given as argument and
takes the inverse power of this value). The Arc on the vector is created
by the line

\begin{verbatim}
# use std::rand;
# use std::sync::Arc;
# fn main() {
# let numbers = Vec::from_fn(1000000, |_| rand::random::<f64>());
let numbers_arc=Arc::new(numbers);
# }
\end{verbatim}

and a unique clone is captured for each task via a procedure. This only
copies the wrapper and not it's contents. Within the task's procedure,
the captured Arc reference can be used as an immutable reference to the
underlying vector as if it were local.

\begin{verbatim}
# use std::rand;
# use std::sync::Arc;
# fn pnorm(nums: &[f64], p: uint) -> f64 { 4.0 }
# fn main() {
# let numbers=Vec::from_fn(1000000, |_| rand::random::<f64>());
# let numbers_arc = Arc::new(numbers);
# let num = 4;
let task_numbers = numbers_arc.clone();
spawn(proc() {
    // Capture task_numbers and use it as if it was the underlying vector
    println!("{}-norm = {}", num, pnorm(task_numbers.as_slice(), num));
});
# }
\end{verbatim}

The \texttt{arc} module also implements Arcs around mutable data that
are not covered here.

\section{Handling task failure}\label{handling-task-failure}

Rust has a built-in mechanism for raising exceptions. The
\texttt{fail!()} macro (which can also be written with an error string
as an argument: \texttt{fail!( \textasciitilde{}reason)}) and the
\texttt{assert!} construct (which effectively calls \texttt{fail!()} if
a boolean expression is false) are both ways to raise exceptions. When a
task raises an exception the task unwinds its stack---running
destructors and freeing memory along the way---and then exits. Unlike
exceptions in C++, exceptions in Rust are unrecoverable within a single
task: once a task fails, there is no way to ``catch'' the exception.

While it isn't possible for a task to recover from failure, tasks may
notify each other of failure. The simplest way of handling task failure
is with the \texttt{try} function, which is similar to \texttt{spawn},
but immediately blocks waiting for the child task to finish.
\texttt{try} returns a value of type
\texttt{Result\textless{}T, ()\textgreater{}}. \texttt{Result} is an
\texttt{enum} type with two variants: \texttt{Ok} and \texttt{Err}. In
this case, because the type arguments to \texttt{Result} are
\texttt{int} and \texttt{()}, callers can pattern-match on a result to
check whether it's an \texttt{Ok} result with an \texttt{int} field
(representing a successful result) or an \texttt{Err} result
(representing termination with an error).

\begin{verbatim}
# use std::task;
# fn some_condition() -> bool { false }
# fn calculate_result() -> int { 0 }
let result: Result<int, ()> = task::try(proc() {
    if some_condition() {
        calculate_result()
    } else {
        fail!("oops!");
    }
});
assert!(result.is_err());
\end{verbatim}

Unlike \texttt{spawn}, the function spawned using \texttt{try} may
return a value, which \texttt{try} will dutifully propagate back to the
caller in a \href{std/result/index.html}{\texttt{Result}} enum. If the
child task terminates successfully, \texttt{try} will return an
\texttt{Ok} result; if the child task fails, \texttt{try} will return an
\texttt{Error} result.

\begin{quote}
\emph{Note:} A failed task does not currently produce a useful error
value (\texttt{try} always returns \texttt{Err(())}). In the future, it
may be possible for tasks to intercept the value passed to
\texttt{fail!()}.
\end{quote}

TODO: Need discussion of \texttt{future\_result} in order to make
failure modes useful.

But not all failures are created equal. In some cases you might need to
abort the entire program (perhaps you're writing an assert which, if it
trips, indicates an unrecoverable logic error); in other cases you might
want to contain the failure at a certain boundary (perhaps a small piece
of input from the outside world, which you happen to be processing in
parallel, is malformed and its processing task can't proceed).

\subsection{Creating a task with a bi-directional communication
path}\label{creating-a-task-with-a-bi-directional-communication-path}

A very common thing to do is to spawn a child task where the parent and
child both need to exchange messages with each other. The function
\texttt{sync::comm::duplex} supports this pattern. We'll look briefly at
how to use it.

To see how \texttt{duplex} works, we will create a child task that
repeatedly receives a \texttt{uint} message, converts it to a string,
and sends the string in response. The child terminates when it receives
\texttt{0}. Here is the function that implements the child task:

\begin{verbatim}
#![allow(deprecated)]

use std::comm::DuplexStream;
# fn main() {
fn stringifier(channel: &DuplexStream<String, uint>) {
    let mut value: uint;
    loop {
        value = channel.recv();
        channel.send(value.to_string());
        if value == 0 { break; }
    }
}
# }
\end{verbatim}

The implementation of \texttt{DuplexStream} supports both sending and
receiving. The \texttt{stringifier} function takes a
\texttt{DuplexStream} that can send strings (the first type parameter)
and receive \texttt{uint} messages (the second type parameter). The body
itself simply loops, reading from the channel and then sending its
response back. The actual response itself is simply the stringified
version of the received value, \texttt{uint::to\_string(value)}.

Here is the code for the parent task:

\begin{verbatim}
#![allow(deprecated)]

use std::comm::duplex;
# use std::task::spawn;
# use std::comm::DuplexStream;
# fn stringifier(channel: &DuplexStream<String, uint>) {
#     let mut value: uint;
#     loop {
#         value = channel.recv();
#         channel.send(value.to_string());
#         if value == 0u { break; }
#     }
# }
# fn main() {

let (from_child, to_child) = duplex();

spawn(proc() {
    stringifier(&to_child);
});

from_child.send(22);
assert!(from_child.recv().as_slice() == "22");

from_child.send(23);
from_child.send(0);

assert!(from_child.recv().as_slice() == "23");
assert!(from_child.recv().as_slice() == "0");

# }
\end{verbatim}

The parent task first calls \texttt{DuplexStream} to create a pair of
bidirectional endpoints. It then uses \texttt{task::spawn} to create the
child task, which captures one end of the communication channel. As a
result, both parent and child can send and receive data to and from the
other.

\end{document}
