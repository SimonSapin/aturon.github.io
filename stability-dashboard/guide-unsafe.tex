\documentclass[]{article}
\usepackage{lmodern}
\usepackage{amssymb,amsmath}
\usepackage{ifxetex,ifluatex}
\usepackage{fixltx2e} % provides \textsubscript
\ifnum 0\ifxetex 1\fi\ifluatex 1\fi=0 % if pdftex
  \usepackage[T1]{fontenc}
  \usepackage[utf8]{inputenc}
\else % if luatex or xelatex
  \ifxetex
    \usepackage{mathspec}
    \usepackage{xltxtra,xunicode}
  \else
    \usepackage{fontspec}
  \fi
  \defaultfontfeatures{Mapping=tex-text,Scale=MatchLowercase}
  \newcommand{\euro}{€}
\fi
% use upquote if available, for straight quotes in verbatim environments
\IfFileExists{upquote.sty}{\usepackage{upquote}}{}
% use microtype if available
\IfFileExists{microtype.sty}{\usepackage{microtype}}{}
\ifxetex
  \usepackage[setpagesize=false, % page size defined by xetex
              unicode=false, % unicode breaks when used with xetex
              xetex]{hyperref}
\else
  \usepackage[unicode=true]{hyperref}
\fi
\hypersetup{breaklinks=true,
            bookmarks=true,
            pdfauthor={},
            pdftitle={Writing Safe Unsafe and Low-Level Code},
            colorlinks=true,
            citecolor=blue,
            urlcolor=blue,
            linkcolor=magenta,
            pdfborder={0 0 0}}
\urlstyle{same}  % don't use monospace font for urls
\setlength{\parindent}{0pt}
\setlength{\parskip}{6pt plus 2pt minus 1pt}
\setlength{\emergencystretch}{3em}  % prevent overfull lines
\setcounter{secnumdepth}{5}

\title{Writing Safe Unsafe and Low-Level Code}

\begin{document}
\maketitle

0.12.0-pre (2c50add48 2014-07-27 23:03:52 -0700)

Copyright © 2011-2014 The Rust Project Developers. Licensed under the
\href{http://www.apache.org/licenses/LICENSE-2.0}{Apache License,
Version 2.0} or the \href{http://opensource.org/licenses/MIT}{MIT
license}, at your option.

This file may not be copied, modified, or distributed except according
to those terms.

{
\hypersetup{linkcolor=black}
\setcounter{tocdepth}{3}
\tableofcontents
}
\section{Introduction}\label{introduction}

Rust aims to provide safe abstractions over the low-level details of the
CPU and operating system, but sometimes one needs to drop down and write
code at that level. This guide aims to provide an overview of the
dangers and power one gets with Rust's unsafe subset.

Rust provides an escape hatch in the form of the
\texttt{unsafe \{ ... \}} block which allows the programmer to dodge
some of the compiler's checks and do a wide range of operations, such
as:

\begin{itemize}
\itemsep1pt\parskip0pt\parsep0pt
\item
  dereferencing \hyperref[raw-pointers]{raw pointers}
\item
  calling a function via FFI (\href{guide-ffi.html}{covered by the FFI
  guide})
\item
  casting between types bitwise (\texttt{transmute}, aka ``reinterpret
  cast'')
\item
  \hyperref[inline-assembly]{inline assembly}
\end{itemize}

Note that an \texttt{unsafe} block does not relax the rules about
lifetimes of \texttt{\&} and the freezing of borrowed data.

Any use of \texttt{unsafe} is the programmer saying ``I know more than
you'' to the compiler, and, as such, the programmer should be very sure
that they actually do know more about why that piece of code is valid.
In general, one should try to minimize the amount of unsafe code in a
code base; preferably by using the bare minimum \texttt{unsafe} blocks
to build safe interfaces.

\begin{quote}
\textbf{Note}: the low-level details of the Rust language are still in
flux, and there is no guarantee of stability or backwards compatibility.
In particular, there may be changes that do not cause compilation
errors, but do cause semantic changes (such as invoking undefined
behaviour). As such, extreme care is required.
\end{quote}

\section{Pointers}\label{pointers}

\subsection{References}\label{references}

One of Rust's biggest features is memory safety. This is achieved in
part via \href{guide-lifetimes.html}{the lifetime system}, which is how
the compiler can guarantee that every \texttt{\&} reference is always
valid, and, for example, never pointing to freed memory.

These restrictions on \texttt{\&} have huge advantages. However, they
also constrain how we can use them. For example, \texttt{\&} doesn't
behave identically to C's pointers, and so cannot be used for pointers
in foreign function interfaces (FFI). Additionally, both immutable
(\texttt{\&}) and mutable (\texttt{\&mut}) references have some aliasing
and freezing guarantees, required for memory safety.

In particular, if you have an \texttt{\&T} reference, then the
\texttt{T} must not be modified through that reference or any other
reference. There are some standard library types, e.g. \texttt{Cell} and
\texttt{RefCell}, that provide inner mutability by replacing compile
time guarantees with dynamic checks at runtime.

An \texttt{\&mut} reference has a different constraint: when an object
has an \texttt{\&mut T} pointing into it, then that \texttt{\&mut}
reference must be the only such usable path to that object in the whole
program. That is, an \texttt{\&mut} cannot alias with any other
references.

Using \texttt{unsafe} code to incorrectly circumvent and violate these
restrictions is undefined behaviour. For example, the following creates
two aliasing \texttt{\&mut} pointers, and is invalid.

\begin{verbatim}
use std::mem;
let mut x: u8 = 1;

let ref_1: &mut u8 = &mut x;
let ref_2: &mut u8 = unsafe { mem::transmute(&mut *ref_1) };

// oops, ref_1 and ref_2 point to the same piece of data (x) and are
// both usable
*ref_1 = 10;
*ref_2 = 20;
\end{verbatim}

\hyperdef{}{raw-pointers}{\subsection{Raw pointers}\label{raw-pointers}}

Rust offers two additional pointer types ``raw pointers'', written as
\texttt{*const T} and \texttt{*mut T}. They're an approximation of C's
\texttt{const T*} and \texttt{T*} respectively; indeed, one of their
most common uses is for FFI, interfacing with external C libraries.

Raw pointers have much fewer guarantees than other pointer types offered
by the Rust language and libraries. For example, they

\begin{itemize}
\itemsep1pt\parskip0pt\parsep0pt
\item
  are not guaranteed to point to valid memory and are not even
  guaranteed to be non-null (unlike both \texttt{Box} and \texttt{\&});
\item
  do not have any automatic clean-up, unlike \texttt{Box}, and so
  require manual resource management;
\item
  are plain-old-data, that is, they don't move ownership, again unlike
  \texttt{Box}, hence the Rust compiler cannot protect against bugs like
  use-after-free;
\item
  are considered sendable (if their contents is considered sendable), so
  the compiler offers no assistance with ensuring their use is
  thread-safe; for example, one can concurrently access a
  \texttt{*mut int} from two threads without synchronization.
\item
  lack any form of lifetimes, unlike \texttt{\&}, and so the compiler
  cannot reason about dangling pointers; and
\item
  have no guarantees about aliasing or mutability other than mutation
  not being allowed directly through a \texttt{*const T}.
\end{itemize}

Fortunately, they come with a redeeming feature: the weaker guarantees
mean weaker restrictions. The missing restrictions make raw pointers
appropriate as a building block for implementing things like smart
pointers and vectors inside libraries. For example, \texttt{*} pointers
are allowed to alias, allowing them to be used to write shared-ownership
types like reference counted and garbage collected pointers, and even
thread-safe shared memory types (\texttt{Rc} and the \texttt{Arc} types
are both implemented entirely in Rust).

There are two things that you are required to be careful about
(i.e.~require an \texttt{unsafe \{ ... \}} block) with raw pointers:

\begin{itemize}
\itemsep1pt\parskip0pt\parsep0pt
\item
  dereferencing: they can have any value: so possible results include a
  crash, a read of uninitialised memory, a use-after-free, or reading
  data as normal.
\item
  pointer arithmetic via the \texttt{offset}
  \hyperref[intrinsics]{intrinsic} (or \texttt{.offset} method): this
  intrinsic uses so-called ``in-bounds'' arithmetic, that is, it is only
  defined behaviour if the result is inside (or one-byte-past-the-end)
  of the object from which the original pointer came.
\end{itemize}

The latter assumption allows the compiler to optimize more effectively.
As can be seen, actually \emph{creating} a raw pointer is not unsafe,
and neither is converting to an integer.

\subsubsection{References and raw
pointers}\label{references-and-raw-pointers}

At runtime, a raw pointer \texttt{*} and a reference pointing to the
same piece of data have an identical representation. In fact, an
\texttt{\&T} reference will implicitly coerce to an \texttt{*const T}
raw pointer in safe code and similarly for the \texttt{mut} variants
(both coercions can be performed explicitly with, respectively,
\texttt{value as *const T} and \texttt{value as *mut T}).

Going the opposite direction, from \texttt{*const} to a reference
\texttt{\&}, is not safe. A \texttt{\&T} is always valid, and so, at a
minimum, the raw pointer \texttt{*const T} has to be a valid to a valid
instance of type \texttt{T}. Furthermore, the resulting pointer must
satisfy the aliasing and mutability laws of references. The compiler
assumes these properties are true for any references, no matter how they
are created, and so any conversion from raw pointers is asserting that
they hold. The programmer \emph{must} guarantee this.

The recommended method for the conversion is

\begin{verbatim}
let i: u32 = 1;
// explicit cast
let p_imm: *const u32 = &i as *const u32;
let mut m: u32 = 2;
// implicit coercion
let p_mut: *mut u32 = &mut m;

unsafe {
    let ref_imm: &u32 = &*p_imm;
    let ref_mut: &mut u32 = &mut *p_mut;
}
\end{verbatim}

The \texttt{\&*x} dereferencing style is preferred to using a
\texttt{transmute}. The latter is far more powerful than necessary, and
the more restricted operation is harder to use incorrectly; for example,
it requires that \texttt{x} is a pointer (unlike \texttt{transmute}).

\subsection{Making the unsafe safe(r)}\label{making-the-unsafe-safer}

There are various ways to expose a safe interface around some unsafe
code:

\begin{itemize}
\itemsep1pt\parskip0pt\parsep0pt
\item
  store pointers privately (i.e.~not in public fields of public
  structs), so that you can see and control all reads and writes to the
  pointer in one place.
\item
  use \texttt{assert!()} a lot: since you can't rely on the protection
  of the compiler \& type-system to ensure that your \texttt{unsafe}
  code is correct at compile-time, use \texttt{assert!()} to verify that
  it is doing the right thing at run-time.
\item
  implement the \texttt{Drop} for resource clean-up via a destructor,
  and use RAII (Resource Acquisition Is Initialization). This reduces
  the need for any manual memory management by users, and automatically
  ensures that clean-up is always run, even when the task fails.
\item
  ensure that any data stored behind a raw pointer is destroyed at the
  appropriate time.
\end{itemize}

As an example, we give a reimplementation of owned boxes by wrapping
\texttt{malloc} and \texttt{free}. Rust's move semantics and lifetimes
mean this reimplementation is as safe as the \texttt{Box} type.

\begin{verbatim}
#![feature(unsafe_destructor)]

extern crate libc;
use libc::{c_void, size_t, malloc, free};
use std::mem;
use std::ptr;

// Define a wrapper around the handle returned by the foreign code.
// Unique<T> has the same semantics as Box<T>
pub struct Unique<T> {
    // It contains a single raw, mutable pointer to the object in question.
    ptr: *mut T
}

// Implement methods for creating and using the values in the box.

// NB: For simplicity and correctness, we require that T has kind Send
// (owned boxes relax this restriction, and can contain managed (GC) boxes).
// This is because, as implemented, the garbage collector would not know
// about any shared boxes stored in the malloc'd region of memory.
impl<T: Send> Unique<T> {
    pub fn new(value: T) -> Unique<T> {
        unsafe {
            let ptr = malloc(mem::size_of::<T>() as size_t) as *mut T;
            // we *need* valid pointer.
            assert!(!ptr.is_null());
            // `*ptr` is uninitialized, and `*ptr = value` would
            // attempt to destroy it `overwrite` moves a value into
            // this memory without attempting to drop the original
            // value.
            ptr::write(&mut *ptr, value);
            Unique{ptr: ptr}
        }
    }

    // the 'r lifetime results in the same semantics as `&*x` with
    // Box<T>
    pub fn borrow<'r>(&'r self) -> &'r T {
        // By construction, self.ptr is valid
        unsafe { &*self.ptr }
    }

    // the 'r lifetime results in the same semantics as `&mut *x` with
    // Box<T>
    pub fn borrow_mut<'r>(&'r mut self) -> &'r mut T {
        unsafe { &mut *self.ptr }
    }
}

// A key ingredient for safety, we associate a destructor with
// Unique<T>, making the struct manage the raw pointer: when the
// struct goes out of scope, it will automatically free the raw pointer.
//
// NB: This is an unsafe destructor, because rustc will not normally
// allow destructors to be associated with parameterized types, due to
// bad interaction with managed boxes. (With the Send restriction,
// we don't have this problem.) Note that the `#[unsafe_destructor]`
// feature gate is required to use unsafe destructors.
#[unsafe_destructor]
impl<T: Send> Drop for Unique<T> {
    fn drop(&mut self) {
        unsafe {
            // Copy the object out from the pointer onto the stack,
            // where it is covered by normal Rust destructor semantics
            // and cleans itself up, if necessary
            ptr::read(self.ptr as *const T);

            // clean-up our allocation
            free(self.ptr as *mut c_void)
        }
    }
}

// A comparison between the built-in `Box` and this reimplementation
fn main() {
    {
        let mut x = box 5i;
        *x = 10;
    } // `x` is freed here

    {
        let mut y = Unique::new(5i);
        *y.borrow_mut() = 10;
    } // `y` is freed here
}
\end{verbatim}

Notably, the only way to construct a \texttt{Unique} is via the
\texttt{new} function, and this function ensures that the internal
pointer is valid and hidden in the private field. The two
\texttt{borrow} methods are safe because the compiler statically
guarantees that objects are never used before creation or after
destruction (unless you use some \texttt{unsafe} code\ldots{}).

\hyperdef{}{inline-assembly}{\section{Inline
assembly}\label{inline-assembly}}

For extremely low-level manipulations and performance reasons, one might
wish to control the CPU directly. Rust supports using inline assembly to
do this via the \texttt{asm!} macro. The syntax roughly matches that of
GCC \& Clang:

\begin{verbatim}
asm!(assembly template
   : output operands
   : input operands
   : clobbers
   : options
   );
\end{verbatim}

Any use of \texttt{asm} is feature gated (requires
\texttt{\#!{[}feature(asm){]}} on the crate to allow) and of course
requires an \texttt{unsafe} block.

\begin{quote}
\textbf{Note}: the examples here are given in x86/x86-64 assembly, but
all platforms are supported.
\end{quote}

\subsection{Assembly template}\label{assembly-template}

The \texttt{assembly template} is the only required parameter and must
be a literal string (i.e \texttt{""})

\begin{verbatim}
#![feature(asm)]

#[cfg(target_arch = "x86")]
#[cfg(target_arch = "x86_64")]
fn foo() {
    unsafe {
        asm!("NOP");
    }
}

// other platforms
#[cfg(not(target_arch = "x86"),
      not(target_arch = "x86_64"))]
fn foo() { /* ... */ }

fn main() {
    // ...
    foo();
    // ...
}
\end{verbatim}

(The \texttt{feature(asm)} and \texttt{\#{[}cfg{]}}s are omitted from
now on.)

Output operands, input operands, clobbers and options are all optional
but you must add the right number of \texttt{:} if you skip them:

\begin{verbatim}
# #![feature(asm)]
# #[cfg(target_arch = "x86")] #[cfg(target_arch = "x86_64")]
# fn main() { unsafe {
asm!("xor %eax, %eax"
    :
    :
    : "eax"
   );
# } }
\end{verbatim}

Whitespace also doesn't matter:

\begin{verbatim}
# #![feature(asm)]
# #[cfg(target_arch = "x86")] #[cfg(target_arch = "x86_64")]
# fn main() { unsafe {
asm!("xor %eax, %eax" ::: "eax");
# } }
\end{verbatim}

\subsection{Operands}\label{operands}

Input and output operands follow the same format:
\texttt{: "constraints1"(expr1), "constraints2"(expr2), ..."}. Output
operand expressions must be mutable lvalues:

\begin{verbatim}
# #![feature(asm)]
# #[cfg(target_arch = "x86")] #[cfg(target_arch = "x86_64")]
fn add(a: int, b: int) -> int {
    let mut c = 0;
    unsafe {
        asm!("add $2, $0"
             : "=r"(c)
             : "0"(a), "r"(b)
             );
    }
    c
}
# #[cfg(not(target_arch = "x86"), not(target_arch = "x86_64"))]
# fn add(a: int, b: int) -> int { a + b }

fn main() {
    assert_eq!(add(3, 14159), 14162)
}
\end{verbatim}

\subsection{Clobbers}\label{clobbers}

Some instructions modify registers which might otherwise have held
different values so we use the clobbers list to indicate to the compiler
not to assume any values loaded into those registers will stay valid.

\begin{verbatim}
# #![feature(asm)]
# #[cfg(target_arch = "x86")] #[cfg(target_arch = "x86_64")]
# fn main() { unsafe {
// Put the value 0x200 in eax
asm!("mov $$0x200, %eax" : /* no outputs */ : /* no inputs */ : "eax");
# } }
\end{verbatim}

Input and output registers need not be listed since that information is
already communicated by the given constraints. Otherwise, any other
registers used either implicitly or explicitly should be listed.

If the assembly changes the condition code register \texttt{cc} should
be specified as one of the clobbers. Similarly, if the assembly modifies
memory, \texttt{memory} should also be specified.

\subsection{Options}\label{options}

The last section, \texttt{options} is specific to Rust. The format is
comma separated literal strings (i.e \texttt{:"foo", "bar", "baz"}).
It's used to specify some extra info about the inline assembly:

Current valid options are:

\begin{enumerate}
\def\labelenumi{\arabic{enumi}.}
\itemsep1pt\parskip0pt\parsep0pt
\item
  \textbf{volatile} - specifying this is analogous to
  \texttt{\_\_asm\_\_ \_\_volatile\_\_ (...)} in gcc/clang.
\item
  \textbf{alignstack} - certain instructions expect the stack to be
  aligned a certain way (i.e SSE) and specifying this indicates to the
  compiler to insert its usual stack alignment code
\item
  \textbf{intel} - use intel syntax instead of the default AT\&T.
\end{enumerate}

\section{Avoiding the standard
library}\label{avoiding-the-standard-library}

By default, \texttt{std} is linked to every Rust crate. In some
contexts, this is undesirable, and can be avoided with the
\texttt{\#!{[}no\_std{]}} attribute attached to the crate.

\begin{verbatim}
// a minimal library
#![crate_type="lib"]
#![no_std]
# // fn main() {} tricked you, rustdoc!
\end{verbatim}

Obviously there's more to life than just libraries: one can use
\texttt{\#{[}no\_std{]}} with an executable, controlling the entry point
is possible in two ways: the \texttt{\#{[}start{]}} attribute, or
overriding the default shim for the C \texttt{main} function with your
own.

The function marked \texttt{\#{[}start{]}} is passed the command line
parameters in the same format as a C:

\begin{verbatim}
#![no_std]
#![feature(lang_items)]

// Pull in the system libc library for what crt0.o likely requires
extern crate libc;

// Entry point for this program
#[start]
fn start(_argc: int, _argv: *const *const u8) -> int {
    0
}

// These functions are invoked by the compiler, but not
// for a bare-bones hello world. These are normally
// provided by libstd.
#[lang = "stack_exhausted"] extern fn stack_exhausted() {}
#[lang = "eh_personality"] extern fn eh_personality() {}
# // fn main() {} tricked you, rustdoc!
\end{verbatim}

To override the compiler-inserted \texttt{main} shim, one has to disable
it with \texttt{\#!{[}no\_main{]}} and then create the appropriate
symbol with the correct ABI and the correct name, which requires
overriding the compiler's name mangling too:

\begin{verbatim}
#![no_std]
#![no_main]
#![feature(lang_items)]

extern crate libc;

#[no_mangle] // ensure that this symbol is called `main` in the output
pub extern fn main(argc: int, argv: *const *const u8) -> int {
    0
}

#[lang = "stack_exhausted"] extern fn stack_exhausted() {}
#[lang = "eh_personality"] extern fn eh_personality() {}
# // fn main() {} tricked you, rustdoc!
\end{verbatim}

The compiler currently makes a few assumptions about symbols which are
available in the executable to call. Normally these functions are
provided by the standard library, but without it you must define your
own.

The first of these two functions, \texttt{stack\_exhausted}, is invoked
whenever stack overflow is detected. This function has a number of
restrictions about how it can be called and what it must do, but if the
stack limit register is not being maintained then a task always has an
``infinite stack'' and this function shouldn't get triggered.

The second of these two functions, \texttt{eh\_personality}, is used by
the failure mechanisms of the compiler. This is often mapped to GCC's
personality function (see the \href{std/rt/unwind/index.html}{libstd
implementation} for more information), but crates which do not trigger
failure can be assured that this function is never called.

\subsection{Using libcore}\label{using-libcore}

\begin{quote}
\textbf{Note}: the core library's structure is unstable, and it is
recommended to use the standard library instead wherever possible.
\end{quote}

With the above techniques, we've got a bare-metal executable running
some Rust code. There is a good deal of functionality provided by the
standard library, however, that is necessary to be productive in Rust.
If the standard library is not sufficient, then
\href{core/index.html}{libcore} is designed to be used instead.

The core library has very few dependencies and is much more portable
than the standard library itself. Additionally, the core library has
most of the necessary functionality for writing idiomatic and effective
Rust code.

As an example, here is a program that will calculate the dot product of
two vectors provided from C, using idiomatic Rust practices.

\begin{verbatim}
#![no_std]
#![feature(globs)]
#![feature(lang_items)]

# extern crate libc;
extern crate core;

use core::prelude::*;

use core::mem;
use core::raw::Slice;

#[no_mangle]
pub extern fn dot_product(a: *const u32, a_len: u32,
                          b: *const u32, b_len: u32) -> u32 {
    // Convert the provided arrays into Rust slices.
    // The core::raw module guarantees that the Slice
    // structure has the same memory layout as a &[T]
    // slice.
    //
    // This is an unsafe operation because the compiler
    // cannot tell the pointers are valid.
    let (a_slice, b_slice): (&[u32], &[u32]) = unsafe {
        mem::transmute((
            Slice { data: a, len: a_len as uint },
            Slice { data: b, len: b_len as uint },
        ))
    };

    // Iterate over the slices, collecting the result
    let mut ret = 0;
    for (i, j) in a_slice.iter().zip(b_slice.iter()) {
        ret += (*i) * (*j);
    }
    return ret;
}

#[lang = "begin_unwind"]
extern fn begin_unwind(args: &core::fmt::Arguments,
                       file: &str,
                       line: uint) -> ! {
    loop {}
}

#[lang = "stack_exhausted"] extern fn stack_exhausted() {}
#[lang = "eh_personality"] extern fn eh_personality() {}
# #[start] fn start(argc: int, argv: *const *const u8) -> int { 0 }
# fn main() {}
\end{verbatim}

Note that there is one extra lang item here which differs from the
examples above, \texttt{begin\_unwind}. This must be defined by
consumers of libcore because the core library declares failure, but it
does not define it. The \texttt{begin\_unwind} lang item is this crate's
definition of failure, and it must be guaranteed to never return.

As can be seen in this example, the core library is intended to provide
the power of Rust in all circumstances, regardless of platform
requirements. Further libraries, such as liballoc, add functionality to
libcore which make other platform-specific assumptions, but continue to
be more portable than the standard library itself.

\section{Interacting with the compiler
internals}\label{interacting-with-the-compiler-internals}

\begin{quote}
\textbf{Note}: this section is specific to the \texttt{rustc} compiler;
these parts of the language may never be full specified and so details
may differ wildly between implementations (and even versions of
\texttt{rustc} itself).

Furthermore, this is just an overview; the best form of documentation
for specific instances of these features are their definitions and uses
in \texttt{std}.
\end{quote}

The Rust language currently has two orthogonal mechanisms for allowing
libraries to interact directly with the compiler and vice versa:

\begin{itemize}
\itemsep1pt\parskip0pt\parsep0pt
\item
  intrinsics, functions built directly into the compiler providing very
  basic low-level functionality,
\item
  lang-items, special functions, types and traits in libraries marked
  with specific \texttt{\#{[}lang{]}} attributes
\end{itemize}

\hyperdef{}{intrinsics}{\subsection{Intrinsics}\label{intrinsics}}

\begin{quote}
\textbf{Note}: intrinsics will forever have an unstable interface, it is
recommended to use the stable interfaces of libcore rather than
intrinsics directly.
\end{quote}

These are imported as if they were FFI functions, with the special
\texttt{rust-intrinsic} ABI. For example, if one was in a freestanding
context, but wished to be able to \texttt{transmute} between types, and
perform efficient pointer arithmetic, one would import those functions
via a declaration like

\begin{verbatim}
# #![feature(intrinsics)]
# fn main() {}

extern "rust-intrinsic" {
    fn transmute<T, U>(x: T) -> U;

    fn offset<T>(dst: *const T, offset: int) -> *const T;
}
\end{verbatim}

As with any other FFI functions, these are always \texttt{unsafe} to
call.

\subsection{Lang items}\label{lang-items}

\begin{quote}
\textbf{Note}: lang items are often provided by crates in the Rust
distribution, and lang items themselves have an unstable interface. It
is recommended to use officially distributed crates instead of defining
your own lang items.
\end{quote}

The \texttt{rustc} compiler has certain pluggable operations, that is,
functionality that isn't hard-coded into the language, but is
implemented in libraries, with a special marker to tell the compiler it
exists. The marker is the attribute \texttt{\#{[}lang="..."{]}} and
there are various different values of \texttt{...}, i.e.~various
different ``lang items''.

For example, \texttt{Box} pointers require two lang items, one for
allocation and one for deallocation. A freestanding program that uses
the \texttt{Box} sugar for dynamic allocations via \texttt{malloc} and
\texttt{free}:

\begin{verbatim}
#![no_std]
#![feature(lang_items)]

extern crate libc;

extern {
    fn abort() -> !;
}

#[lang="exchange_malloc"]
unsafe fn allocate(size: uint, _align: uint) -> *mut u8 {
    let p = libc::malloc(size as libc::size_t) as *mut u8;

    // malloc failed
    if p as uint == 0 {
        abort();
    }

    p
}
#[lang="exchange_free"]
unsafe fn deallocate(ptr: *mut u8, _size: uint, _align: uint) {
    libc::free(ptr as *mut libc::c_void)
}

#[start]
fn main(argc: int, argv: *const *const u8) -> int {
    let x = box 1i;

    0
}

#[lang = "stack_exhausted"] extern fn stack_exhausted() {}
#[lang = "eh_personality"] extern fn eh_personality() {}
\end{verbatim}

Note the use of \texttt{abort}: the \texttt{exchange\_malloc} lang item
is assumed to return a valid pointer, and so needs to do the check
internally.

Other features provided by lang items include:

\begin{itemize}
\itemsep1pt\parskip0pt\parsep0pt
\item
  overloadable operators via traits: the traits corresponding to the
  \texttt{==}, \texttt{\textless{}}, dereferencing (\texttt{*}) and
  \texttt{+} (etc.) operators are all marked with lang items; those
  specific four are \texttt{eq}, \texttt{ord}, \texttt{deref}, and
  \texttt{add} respectively.
\item
  stack unwinding and general failure; the \texttt{eh\_personality},
  \texttt{fail\_} and \texttt{fail\_bounds\_checks} lang items.
\item
  the traits in \texttt{std::kinds} used to indicate types that satisfy
  various kinds; lang items \texttt{send}, \texttt{share} and
  \texttt{copy}.
\item
  the marker types and variance indicators found in
  \texttt{std::kinds::markers}; lang items \texttt{covariant\_type},
  \texttt{contravariant\_lifetime}, \texttt{no\_share\_bound}, etc.
\end{itemize}

Lang items are loaded lazily by the compiler; e.g.~if one never uses
\texttt{Box} then there is no need to define functions for
\texttt{exchange\_malloc} and \texttt{exchange\_free}. \texttt{rustc}
will emit an error when an item is needed but not found in the current
crate or any that it depends on.

\end{document}
