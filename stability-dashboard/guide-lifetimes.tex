\documentclass[]{article}
\usepackage{lmodern}
\usepackage{amssymb,amsmath}
\usepackage{ifxetex,ifluatex}
\usepackage{fixltx2e} % provides \textsubscript
\ifnum 0\ifxetex 1\fi\ifluatex 1\fi=0 % if pdftex
  \usepackage[T1]{fontenc}
  \usepackage[utf8]{inputenc}
\else % if luatex or xelatex
  \ifxetex
    \usepackage{mathspec}
    \usepackage{xltxtra,xunicode}
  \else
    \usepackage{fontspec}
  \fi
  \defaultfontfeatures{Mapping=tex-text,Scale=MatchLowercase}
  \newcommand{\euro}{€}
\fi
% use upquote if available, for straight quotes in verbatim environments
\IfFileExists{upquote.sty}{\usepackage{upquote}}{}
% use microtype if available
\IfFileExists{microtype.sty}{\usepackage{microtype}}{}
\ifxetex
  \usepackage[setpagesize=false, % page size defined by xetex
              unicode=false, % unicode breaks when used with xetex
              xetex]{hyperref}
\else
  \usepackage[unicode=true]{hyperref}
\fi
\hypersetup{breaklinks=true,
            bookmarks=true,
            pdfauthor={},
            pdftitle={The Rust References and Lifetimes Guide},
            colorlinks=true,
            citecolor=blue,
            urlcolor=blue,
            linkcolor=magenta,
            pdfborder={0 0 0}}
\urlstyle{same}  % don't use monospace font for urls
\setlength{\parindent}{0pt}
\setlength{\parskip}{6pt plus 2pt minus 1pt}
\setlength{\emergencystretch}{3em}  % prevent overfull lines
\setcounter{secnumdepth}{5}

\title{The Rust References and Lifetimes Guide}

\begin{document}
\maketitle

0.11.0 (305cf1f1098dc87e346003e6bf7e7cca5705a135 2014-07-10 11:19:20 -0700)

Copyright © 2011-2014 The Rust Project Developers. Licensed under the
\href{http://www.apache.org/licenses/LICENSE-2.0}{Apache License,
Version 2.0} or the \href{http://opensource.org/licenses/MIT}{MIT
license}, at your option.

This file may not be copied, modified, or distributed except according
to those terms.

{
\hypersetup{linkcolor=black}
\setcounter{tocdepth}{3}
\tableofcontents
}
\section{Introduction}\label{introduction}

References are one of the more flexible and powerful tools available in
Rust. They can point anywhere: into the heap, stack, and even into the
interior of another data structure. A reference is as flexible as a C
pointer or C++ reference.

Unlike C and C++ compilers, the Rust compiler includes special static
checks that ensure that programs use references safely.

Despite their complete safety, a reference's representation at runtime
is the same as that of an ordinary pointer in a C program. They
introduce zero overhead. The compiler does all safety checks at compile
time.

Although references have rather elaborate theoretical underpinnings
(e.g.~region pointers), the core concepts will be familiar to anyone who
has worked with C or C++. The best way to explain how they are
used---and their limitations---is probably just to work through several
examples.

\section{By example}\label{by-example}

References, sometimes known as \emph{borrowed pointers}, are only valid
for a limited duration. References never claim any kind of ownership
over the data that they point to. Instead, they are used for cases where
you would like to use data for a short time.

Consider a simple struct type \texttt{Point}:

\begin{verbatim}
struct Point {x: f64, y: f64}
\end{verbatim}

We can use this simple definition to allocate points in many different
ways. For example, in this code, each of these local variables contains
a point, but allocated in a different place:

\begin{verbatim}
# struct Point {x: f64, y: f64}
let on_the_stack : Point      =     Point {x: 3.0, y: 4.0};
let on_the_heap  : Box<Point> = box Point {x: 7.0, y: 9.0};
\end{verbatim}

Suppose we wanted to write a procedure that computed the distance
between any two points, no matter where they were stored. One option is
to define a function that takes two arguments of type
\texttt{Point}---that is, it takes the points by value. But if we define
it this way, calling the function will cause the points to be copied.
For points, this is probably not so bad, but often copies are expensive.
So we'd like to define a function that takes the points just as a
reference.

\begin{verbatim}
# struct Point {x: f64, y: f64}
# fn sqrt(f: f64) -> f64 { 0.0 }
fn compute_distance(p1: &Point, p2: &Point) -> f64 {
    let x_d = p1.x - p2.x;
    let y_d = p1.y - p2.y;
    sqrt(x_d * x_d + y_d * y_d)
}
\end{verbatim}

Now we can call \texttt{compute\_distance()}:

\begin{verbatim}
# struct Point {x: f64, y: f64}
# let on_the_stack :     Point  =     Point{x: 3.0, y: 4.0};
# let on_the_heap  : Box<Point> = box Point{x: 7.0, y: 9.0};
# fn compute_distance(p1: &Point, p2: &Point) -> f64 { 0.0 }
compute_distance(&on_the_stack, on_the_heap);
\end{verbatim}

Here, the \texttt{\&} operator takes the address of the variable
\texttt{on\_the\_stack}; this is because \texttt{on\_the\_stack} has the
type \texttt{Point} (that is, a struct value) and we have to take its
address to get a value. We also call this \emph{borrowing} the local
variable \texttt{on\_the\_stack}, because we have created an alias: that
is, another name for the same data.

In the case of \texttt{on\_the\_heap}, however, no explicit action is
necessary. The compiler will automatically convert a box point to a
reference like \&point. This is another form of borrowing; in this case,
the contents of the owned box are being lent out.

Whenever a caller lends data to a callee, there are some limitations on
what the caller can do with the original. For example, if the contents
of a variable have been lent out, you cannot send that variable to
another task. In addition, the compiler will reject any code that might
cause the borrowed value to be freed or overwrite its component fields
with values of different types (I'll get into what kinds of actions
those are shortly). This rule should make intuitive sense: you must wait
for a borrower to return the value that you lent it (that is, wait for
the reference to go out of scope) before you can make full use of it
again.

\section{Other uses for the \&
operator}\label{other-uses-for-the-operator}

In the previous example, the value \texttt{on\_the\_stack} was defined
like so:

\begin{verbatim}
# struct Point {x: f64, y: f64}
let on_the_stack: Point = Point {x: 3.0, y: 4.0};
\end{verbatim}

This declaration means that code can only pass \texttt{Point} by value
to other functions. As a consequence, we had to explicitly take the
address of \texttt{on\_the\_stack} to get a reference. Sometimes however
it is more convenient to move the \& operator into the definition of
\texttt{on\_the\_stack}:

\begin{verbatim}
# struct Point {x: f64, y: f64}
let on_the_stack2: &Point = &Point {x: 3.0, y: 4.0};
\end{verbatim}

Applying \texttt{\&} to an rvalue (non-assignable location) is just a
convenient shorthand for creating a temporary and taking its address. A
more verbose way to write the same code is:

\begin{verbatim}
# struct Point {x: f64, y: f64}
let tmp = Point {x: 3.0, y: 4.0};
let on_the_stack2 : &Point = &tmp;
\end{verbatim}

\section{Taking the address of
fields}\label{taking-the-address-of-fields}

The \texttt{\&} operator is not limited to taking the address of local
variables. It can also take the address of fields or individual array
elements. For example, consider this type definition for
\texttt{Rectangle}:

\begin{verbatim}
struct Point {x: f64, y: f64} // as before
struct Size {w: f64, h: f64} // as before
struct Rectangle {origin: Point, size: Size}
\end{verbatim}

Now, as before, we can define rectangles in a few different ways:

\begin{verbatim}
# struct Point {x: f64, y: f64}
# struct Size {w: f64, h: f64} // as before
# struct Rectangle {origin: Point, size: Size}
let rect_stack   =    &Rectangle {origin: Point {x: 1.0, y: 2.0},
                                  size: Size {w: 3.0, h: 4.0}};
let rect_heap    = box Rectangle {origin: Point {x: 5.0, y: 6.0},
                                  size: Size {w: 3.0, h: 4.0}};
\end{verbatim}

In each case, we can extract out individual subcomponents with the
\texttt{\&} operator. For example, I could write:

\begin{verbatim}
# struct Point {x: f64, y: f64} // as before
# struct Size {w: f64, h: f64} // as before
# struct Rectangle {origin: Point, size: Size}
# let rect_stack  = &Rectangle {origin: Point {x: 1.0, y: 2.0}, size: Size {w: 3.0, h: 4.0}};
# let rect_heap   = box Rectangle {origin: Point {x: 5.0, y: 6.0}, size: Size {w: 3.0, h: 4.0}};
# fn compute_distance(p1: &Point, p2: &Point) -> f64 { 0.0 }
compute_distance(&rect_stack.origin, &rect_heap.origin);
\end{verbatim}

which would borrow the field \texttt{origin} from the rectangle on the
stack as well as from the owned box, and then compute the distance
between them.

\section{Lifetimes}\label{lifetimes}

We've seen a few examples of borrowing data. To this point, we've
glossed over issues of safety. As stated in the introduction, at runtime
a reference is simply a pointer, nothing more. Therefore, avoiding C's
problems with dangling pointers requires a compile-time safety check.

The basis for the check is the notion of \emph{lifetimes}. A lifetime is
a static approximation of the span of execution during which the pointer
is valid: it always corresponds to some expression or block within the
program.

The compiler will only allow a borrow \emph{if it can guarantee that the
data will not be reassigned or moved for the lifetime of the pointer}.
This does not necessarily mean that the data is stored in immutable
memory. For example, the following function is legal:

\begin{verbatim}
# fn some_condition() -> bool { true }
# struct Foo { f: int }
fn example3() -> int {
    let mut x = box Foo {f: 3};
    if some_condition() {
        let y = &x.f;      // -+ L
        return *y;         //  |
    }                      // -+
    x = box Foo {f: 4};
    // ...
# return 0;
}
\end{verbatim}

Here, the interior of the variable \texttt{x} is being borrowed and
\texttt{x} is declared as mutable. However, the compiler can prove that
\texttt{x} is not assigned anywhere in the lifetime L of the variable
\texttt{y}. Therefore, it accepts the function, even though \texttt{x}
is mutable and in fact is mutated later in the function.

It may not be clear why we are so concerned about mutating a borrowed
variable. The reason is that the runtime system frees any box \emph{as
soon as its owning reference changes or goes out of scope}. Therefore, a
program like this is illegal (and would be rejected by the compiler):

\begin{verbatim}
fn example3() -> int {
    let mut x = box X {f: 3};
    let y = &x.f;
    x = box X {f: 4};  // Error reported here.
    *y
}
\end{verbatim}

To make this clearer, consider this diagram showing the state of memory
immediately before the re-assignment of \texttt{x}:

\begin{verbatim}
    Stack               Exchange Heap

  x +-------------+
    | box {f:int} | ----+
  y +-------------+     |
    | &int        | ----+
    +-------------+     |    +---------+
                        +--> |  f: 3   |
                             +---------+
\end{verbatim}

Once the reassignment occurs, the memory will look like this:

\begin{verbatim}
    Stack               Exchange Heap

  x +-------------+          +---------+
    | box {f:int} | -------> |  f: 4   |
  y +-------------+          +---------+
    | &int        | ----+
    +-------------+     |    +---------+
                        +--> | (freed) |
                             +---------+
\end{verbatim}

Here you can see that the variable \texttt{y} still points at the old
\texttt{f} property of Foo, which has been freed.

In fact, the compiler can apply the same kind of reasoning to any memory
that is (uniquely) owned by the stack frame. So we could modify the
previous example to introduce additional owned pointers and structs, and
the compiler will still be able to detect possible mutations. This time,
we'll use an analogy to illustrate the concept.

\begin{verbatim}
fn example3() -> int {
    struct House { owner: Box<Person> }
    struct Person { age: int }

    let mut house = box House {
        owner: box Person {age: 30}
    };

    let owner_age = &house.owner.age;
    house = box House {owner: box Person {age: 40}};  // Error reported here.
    house.owner = box Person {age: 50};               // Error reported here.
    *owner_age
}
\end{verbatim}

In this case, two errors are reported, one when the variable
\texttt{house} is modified and another when \texttt{house.owner} is
modified. Either modification would invalidate the pointer
\texttt{owner\_age}.

\section{Borrowing and enums}\label{borrowing-and-enums}

The previous example showed that the type system forbids any mutations
of owned boxed values while they are being borrowed. In general, the
type system also forbids borrowing a value as mutable if it is already
being borrowed - either as a mutable reference or an immutable one. This
restriction prevents pointers from pointing into freed memory. There is
one other case where the compiler must be very careful to ensure that
pointers remain valid: pointers into the interior of an \texttt{enum}.

Let's look at the following \texttt{shape} type that can represent both
rectangles and circles:

\begin{verbatim}
struct Point {x: f64, y: f64}; // as before
struct Size {w: f64, h: f64}; // as before
enum Shape {
    Circle(Point, f64),   // origin, radius
    Rectangle(Point, Size)  // upper-left, dimensions
}
\end{verbatim}

Now we might write a function to compute the area of a shape. This
function takes a reference to a shape, to avoid the need for copying.

\begin{verbatim}
# struct Point {x: f64, y: f64}; // as before
# struct Size {w: f64, h: f64}; // as before
# enum Shape {
#     Circle(Point, f64),   // origin, radius
#     Rectangle(Point, Size)  // upper-left, dimensions
# }
# static tau: f64 = 6.28;
fn compute_area(shape: &Shape) -> f64 {
    match *shape {
        Circle(_, radius) => 0.5 * tau * radius * radius,
        Rectangle(_, ref size) => size.w * size.h
    }
}
\end{verbatim}

The first case matches against circles. Here, the pattern extracts the
radius from the shape variant and the action uses it to compute the area
of the circle. (Like any up-to-date engineer, we use the
\href{http://www.math.utah.edu/~palais/pi.html}{tau circle constant} and
not that dreadfully outdated notion of pi).

The second match is more interesting. Here we match against a rectangle
and extract its size: but rather than copy the \texttt{size} struct, we
use a by-reference binding to create a pointer to it. In other words, a
pattern binding like \texttt{ref size} binds the name \texttt{size} to a
pointer of type \texttt{\&size} into the \emph{interior of the enum}.

To make this more clear, let's look at a diagram of memory layout in the
case where \texttt{shape} points at a rectangle:

\begin{verbatim}
Stack             Memory

+-------+         +---------------+
| shape | ------> | rectangle(    |
+-------+         |   {x: f64,    |
| size  | -+      |    y: f64},   |
+-------+  +----> |   {w: f64,    |
                  |    h: f64})   |
                  +---------------+
\end{verbatim}

Here you can see that rectangular shapes are composed of five words of
memory. The first is a tag indicating which variant this enum is
(\texttt{rectangle}, in this case). The next two words are the
\texttt{x} and \texttt{y} fields for the point and the remaining two are
the \texttt{w} and \texttt{h} fields for the size. The binding
\texttt{size} is then a pointer into the inside of the shape.

Perhaps you can see where the danger lies: if the shape were somehow to
be reassigned, perhaps to a circle, then although the memory used to
store that shape value would still be valid, \emph{it would have a
different type}! The following diagram shows what memory would look like
if code overwrote \texttt{shape} with a circle:

\begin{verbatim}
Stack             Memory

+-------+         +---------------+
| shape | ------> | circle(       |
+-------+         |   {x: f64,    |
| size  | -+      |    y: f64},   |
+-------+  +----> |   f64)        |
                  |               |
                  +---------------+
\end{verbatim}

As you can see, the \texttt{size} pointer would be pointing at a
\texttt{f64} instead of a struct. This is not good: dereferencing the
second field of a \texttt{f64} as if it were a struct with two fields
would be a memory safety violation.

So, in fact, for every \texttt{ref} binding, the compiler will impose
the same rules as the ones we saw for borrowing the interior of an owned
box: it must be able to guarantee that the \texttt{enum} will not be
overwritten for the duration of the borrow. In fact, the compiler would
accept the example we gave earlier. The example is safe because the
shape pointer has type \texttt{\&Shape}, which means ``reference to
immutable memory containing a \texttt{shape}''. If, however, the type of
that pointer were \texttt{\&mut Shape}, then the ref binding would be
ill-typed. Just as with owned boxes, the compiler will permit
\texttt{ref} bindings into data owned by the stack frame even if the
data are mutable, but otherwise it requires that the data reside in
immutable memory.

\section{Returning references}\label{returning-references}

So far, all of the examples we have looked at, use references in a
``downward'' direction. That is, a method or code block creates a
reference, then uses it within the same scope. It is also possible to
return references as the result of a function, but as we'll see, doing
so requires some explicit annotation.

We could write a subroutine like this:

\begin{verbatim}
struct Point {x: f64, y: f64}
fn get_x<'r>(p: &'r Point) -> &'r f64 { &p.x }
\end{verbatim}

Here, the function \texttt{get\_x()} returns a pointer into the
structure it was given. The type of the parameter (\texttt{\&'r Point})
and return type (\texttt{\&'r f64}) both use a new syntactic form that
we have not seen so far. Here the identifier \texttt{r} names the
lifetime of the pointer explicitly. So in effect, this function declares
that it takes a pointer with lifetime \texttt{r} and returns a pointer
with that same lifetime.

In general, it is only possible to return references if they are derived
from a parameter to the procedure. In that case, the pointer result will
always have the same lifetime as one of the parameters; named lifetimes
indicate which parameter that is.

In the previous code samples, function parameter types did not include a
lifetime name. The compiler simply creates a fresh name for the lifetime
automatically: that is, the lifetime name is guaranteed to refer to a
distinct lifetime from the lifetimes of all other parameters.

Named lifetimes that appear in function signatures are conceptually the
same as the other lifetimes we have seen before, but they are a bit
abstract: they don't refer to a specific expression within
\texttt{get\_x()}, but rather to some expression within the \emph{caller
of \texttt{get\_x()}}. The lifetime \texttt{r} is actually a kind of
\emph{lifetime parameter}: it is defined by the caller to
\texttt{get\_x()}, just as the value for the parameter \texttt{p} is
defined by that caller.

In any case, whatever the lifetime of \texttt{r} is, the pointer
produced by \texttt{\&p.x} always has the same lifetime as \texttt{p}
itself: a pointer to a field of a struct is valid as long as the struct
is valid. Therefore, the compiler accepts the function
\texttt{get\_x()}.

To emphasize this point, let's look at a variation on the example, this
time one that does not compile:

\begin{verbatim}
struct Point {x: f64, y: f64}
fn get_x_sh(p: &Point) -> &f64 {
    &p.x // Error reported here
}
\end{verbatim}

Here, the function \texttt{get\_x\_sh()} takes a reference as input and
returns a reference. As before, the lifetime of the reference that will
be returned is a parameter (specified by the caller). That means that
\texttt{get\_x\_sh()} promises to return a reference that is valid for
as long as the caller would like: this is subtly different from the
first example, which promised to return a pointer that was valid for as
long as its pointer argument was valid.

Within \texttt{get\_x\_sh()}, we see the expression \texttt{\&p.x} which
takes the address of a field of a Point. The presence of this expression
implies that the compiler must guarantee that , so long as the resulting
pointer is valid, the original Point won't be moved or changed.

But recall that \texttt{get\_x\_sh()} also promised to return a pointer
that was valid for as long as the caller wanted it to be. Clearly,
\texttt{get\_x\_sh()} is not in a position to make both of these
guarantees; in fact, it cannot guarantee that the pointer will remain
valid at all once it returns, as the parameter \texttt{p} may or may not
be live in the caller. Therefore, the compiler will report an error
here.

In general, if you borrow a struct or box to create a reference, it will
only be valid within the function and cannot be returned. This is why
the typical way to return references is to take references as input (the
only other case in which it can be legal to return a reference is if it
points at a static constant).

\section{Named lifetimes}\label{named-lifetimes}

Lifetimes can be named and referenced. For example, the special lifetime
\texttt{'static}, which does not go out of scope, can be used to create
global variables and communicate between tasks (see the manual for use
cases).

\subsection{Parameter Lifetimes}\label{parameter-lifetimes}

Named lifetimes allow for grouping of parameters by lifetime. For
example, consider this function:

\begin{verbatim}
# struct Point {x: f64, y: f64}; // as before
# struct Size {w: f64, h: f64}; // as before
# enum Shape {
#     Circle(Point, f64),   // origin, radius
#     Rectangle(Point, Size)  // upper-left, dimensions
# }
# fn compute_area(shape: &Shape) -> f64 { 0.0 }
fn select<'r, T>(shape: &'r Shape, threshold: f64,
                 a: &'r T, b: &'r T) -> &'r T {
    if compute_area(shape) > threshold {a} else {b}
}
\end{verbatim}

This function takes three references and assigns each the same lifetime
\texttt{r}. In practice, this means that, in the caller, the lifetime
\texttt{r} will be the \emph{intersection of the lifetime of the three
region parameters}. This may be overly conservative, as in this example:

\begin{verbatim}
# struct Point {x: f64, y: f64}; // as before
# struct Size {w: f64, h: f64}; // as before
# enum Shape {
#     Circle(Point, f64),   // origin, radius
#     Rectangle(Point, Size)  // upper-left, dimensions
# }
# fn compute_area(shape: &Shape) -> f64 { 0.0 }
# fn select<'r, T>(shape: &Shape, threshold: f64,
#                  a: &'r T, b: &'r T) -> &'r T {
#     if compute_area(shape) > threshold {a} else {b}
# }
                                                     // -+ r
fn select_based_on_unit_circle<'r, T>(               //  |-+ B
    threshold: f64, a: &'r T, b: &'r T) -> &'r T {   //  | |
                                                     //  | |
    let shape = Circle(Point {x: 0., y: 0.}, 1.);    //  | |
    select(&shape, threshold, a, b)                  //  | |
}                                                    //  |-+
                                                     // -+
\end{verbatim}

In this call to \texttt{select()}, the lifetime of the first parameter
shape is B, the function body. Both of the second two parameters
\texttt{a} and \texttt{b} share the same lifetime, \texttt{r}, which is
a lifetime parameter of \texttt{select\_based\_on\_unit\_circle()}. The
caller will infer the intersection of these two lifetimes as the
lifetime of the returned value, and hence the return value of
\texttt{select()} will be assigned a lifetime of B. This will in turn
lead to a compilation error, because
\texttt{select\_based\_on\_unit\_circle()} is supposed to return a value
with the lifetime \texttt{r}.

To address this, we can modify the definition of \texttt{select()} to
distinguish the lifetime of the first parameter from the lifetime of the
latter two. After all, the first parameter is not being returned. Here
is how the new \texttt{select()} might look:

\begin{verbatim}
# struct Point {x: f64, y: f64}; // as before
# struct Size {w: f64, h: f64}; // as before
# enum Shape {
#     Circle(Point, f64),   // origin, radius
#     Rectangle(Point, Size)  // upper-left, dimensions
# }
# fn compute_area(shape: &Shape) -> f64 { 0.0 }
fn select<'r, 'tmp, T>(shape: &'tmp Shape, threshold: f64,
                       a: &'r T, b: &'r T) -> &'r T {
    if compute_area(shape) > threshold {a} else {b}
}
\end{verbatim}

Here you can see that \texttt{shape}'s lifetime is now named
\texttt{tmp}. The parameters \texttt{a}, \texttt{b}, and the return
value all have the lifetime \texttt{r}. However, since the lifetime
\texttt{tmp} is not returned, it would be more concise to just omit the
named lifetime for \texttt{shape} altogether:

\begin{verbatim}
# struct Point {x: f64, y: f64}; // as before
# struct Size {w: f64, h: f64}; // as before
# enum Shape {
#     Circle(Point, f64),   // origin, radius
#     Rectangle(Point, Size)  // upper-left, dimensions
# }
# fn compute_area(shape: &Shape) -> f64 { 0.0 }
fn select<'r, T>(shape: &Shape, threshold: f64,
                 a: &'r T, b: &'r T) -> &'r T {
    if compute_area(shape) > threshold {a} else {b}
}
\end{verbatim}

This is equivalent to the previous definition.

\subsection{Labeled Control
Structures}\label{labeled-control-structures}

Named lifetime notation can also be used to control the flow of
execution:

\begin{verbatim}
'h: for i in range(0u, 10) {
    'g: loop {
        if i % 2 == 0 { continue 'h; }
        if i == 9 { break 'h; }
        break 'g;
    }
}
\end{verbatim}

\begin{quote}
\emph{Note:} Labelled breaks are not currently supported within
\texttt{while} loops.
\end{quote}

Named labels are hygienic and can be used safely within macros. See the
macros guide section on hygiene for more details.

\section{Conclusion}\label{conclusion}

So there you have it: a (relatively) brief tour of the lifetime system.
For more details, we refer to the (yet to be written) reference document
on references, which will explain the full notation and give more
examples.

\end{document}
