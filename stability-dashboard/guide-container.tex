\documentclass[]{article}
\usepackage{lmodern}
\usepackage{amssymb,amsmath}
\usepackage{ifxetex,ifluatex}
\usepackage{fixltx2e} % provides \textsubscript
\ifnum 0\ifxetex 1\fi\ifluatex 1\fi=0 % if pdftex
  \usepackage[T1]{fontenc}
  \usepackage[utf8]{inputenc}
\else % if luatex or xelatex
  \ifxetex
    \usepackage{mathspec}
    \usepackage{xltxtra,xunicode}
  \else
    \usepackage{fontspec}
  \fi
  \defaultfontfeatures{Mapping=tex-text,Scale=MatchLowercase}
  \newcommand{\euro}{€}
\fi
% use upquote if available, for straight quotes in verbatim environments
\IfFileExists{upquote.sty}{\usepackage{upquote}}{}
% use microtype if available
\IfFileExists{microtype.sty}{\usepackage{microtype}}{}
\ifxetex
  \usepackage[setpagesize=false, % page size defined by xetex
              unicode=false, % unicode breaks when used with xetex
              xetex]{hyperref}
\else
  \usepackage[unicode=true]{hyperref}
\fi
\hypersetup{breaklinks=true,
            bookmarks=true,
            pdfauthor={},
            pdftitle={The Rust Containers and Iterators Guide},
            colorlinks=true,
            citecolor=blue,
            urlcolor=blue,
            linkcolor=magenta,
            pdfborder={0 0 0}}
\urlstyle{same}  % don't use monospace font for urls
\setlength{\parindent}{0pt}
\setlength{\parskip}{6pt plus 2pt minus 1pt}
\setlength{\emergencystretch}{3em}  % prevent overfull lines
\setcounter{secnumdepth}{5}

\title{The Rust Containers and Iterators Guide}

\begin{document}
\maketitle

0.12.0-pre (2c50add48 2014-07-27 23:03:52 -0700)

Copyright © 2011-2014 The Rust Project Developers. Licensed under the
\href{http://www.apache.org/licenses/LICENSE-2.0}{Apache License,
Version 2.0} or the \href{http://opensource.org/licenses/MIT}{MIT
license}, at your option.

This file may not be copied, modified, or distributed except according
to those terms.

{
\hypersetup{linkcolor=black}
\setcounter{tocdepth}{3}
\tableofcontents
}
\section{Containers}\label{containers}

The container traits are defined in the \texttt{std::container} module.

\subsection{Unique vectors}\label{unique-vectors}

Vectors have \texttt{O(1)} indexing, push (to the end) and pop (from the
end). Vectors are the most common container in Rust, and are flexible
enough to fit many use cases.

Vectors can also be sorted and used as efficient lookup tables with the
\texttt{bsearch()} method, if all the elements are inserted at one time
and deletions are unnecessary.

\subsection{Maps and sets}\label{maps-and-sets}

Maps are collections of unique keys with corresponding values, and sets
are just unique keys without a corresponding value. The \texttt{Map} and
\texttt{Set} traits in \texttt{std::container} define the basic
interface.

The standard library provides three owned map/set types:

\begin{itemize}
\itemsep1pt\parskip0pt\parsep0pt
\item
  \texttt{collections::HashMap} and \texttt{collections::HashSet},
  requiring the keys to implement \texttt{Eq} and \texttt{Hash}
\item
  \texttt{collections::TrieMap} and \texttt{collections::TrieSet},
  requiring the keys to be \texttt{uint}
\item
  \texttt{collections::TreeMap} and \texttt{collections::TreeSet},
  requiring the keys to implement \texttt{Ord}
\end{itemize}

These maps do not use managed pointers so they can be sent between tasks
as long as the key and value types are sendable. Neither the key or
value type has to be copyable.

The \texttt{TrieMap} and \texttt{TreeMap} maps are ordered, while
\texttt{HashMap} uses an arbitrary order.

Each \texttt{HashMap} instance has a random 128-bit key to use with a
keyed hash, making the order of a set of keys in a given hash table
randomized. Rust provides a \href{https://131002.net/siphash/}{SipHash}
implementation for any type implementing the \texttt{Hash} trait.

\subsection{Double-ended queues}\label{double-ended-queues}

The \texttt{collections::ringbuf} module implements a double-ended queue
with \texttt{O(1)} amortized inserts and removals from both ends of the
container. It also has \texttt{O(1)} indexing like a vector. The
contained elements are not required to be copyable, and the queue will
be sendable if the contained type is sendable. Its interface
\texttt{Deque} is defined in \texttt{collections}.

The \texttt{extra::dlist} module implements a double-ended linked list,
also implementing the \texttt{Deque} trait, with \texttt{O(1)} removals
and inserts at either end, and \texttt{O(1)} concatenation.

\subsection{Priority queues}\label{priority-queues}

The \texttt{collections::priority\_queue} module implements a queue
ordered by a key. The contained elements are not required to be
copyable, and the queue will be sendable if the contained type is
sendable.

Insertions have \texttt{O(log n)} time complexity and checking or
popping the largest element is \texttt{O(1)}. Converting a vector to a
priority queue can be done in-place, and has \texttt{O(n)} complexity. A
priority queue can also be converted to a sorted vector in-place,
allowing it to be used for an \texttt{O(n log n)} in-place heapsort.

\section{Iterators}\label{iterators}

\subsection{Iteration protocol}\label{iteration-protocol}

The iteration protocol is defined by the \texttt{Iterator} trait in the
\texttt{std::iter} module. The minimal implementation of the trait is a
\texttt{next} method, yielding the next element from an iterator object:

\begin{verbatim}
/// An infinite stream of zeroes
struct ZeroStream;

impl Iterator<int> for ZeroStream {
    fn next(&mut self) -> Option<int> {
        Some(0)
    }
}
\end{verbatim}

Reaching the end of the iterator is signalled by returning \texttt{None}
instead of \texttt{Some(item)}:

\begin{verbatim}
# fn main() {}
/// A stream of N zeroes
struct ZeroStream {
    remaining: uint
}

impl ZeroStream {
    fn new(n: uint) -> ZeroStream {
        ZeroStream { remaining: n }
    }
}

impl Iterator<int> for ZeroStream {
    fn next(&mut self) -> Option<int> {
        if self.remaining == 0 {
            None
        } else {
            self.remaining -= 1;
            Some(0)
        }
    }
}
\end{verbatim}

In general, you cannot rely on the behavior of the \texttt{next()}
method after it has returned \texttt{None}. Some iterators may return
\texttt{None} forever. Others may behave differently.

\subsection{Container iterators}\label{container-iterators}

Containers implement iteration over the contained elements by returning
an iterator object. For example, for vector slices several iterators are
available:

\begin{itemize}
\itemsep1pt\parskip0pt\parsep0pt
\item
  \texttt{iter()} for immutable references to the elements
\item
  \texttt{mut\_iter()} for mutable references to the elements
\item
  \texttt{move\_iter()} to move the elements out by-value
\end{itemize}

A typical mutable container will implement at least \texttt{iter()},
\texttt{mut\_iter()} and \texttt{move\_iter()}. If it maintains an
order, the returned iterators will be \texttt{DoubleEndedIterator}s,
which are described below.

\subsubsection{Freezing}\label{freezing}

Unlike most other languages with external iterators, Rust has no
\emph{iterator invalidation}. As long as an iterator is still in scope,
the compiler will prevent modification of the container through another
handle.

\begin{verbatim}
let mut xs = [1i, 2, 3];
{
    let _it = xs.iter();

    // the vector is frozen for this scope, the compiler will statically
    // prevent modification
}
// the vector becomes unfrozen again at the end of the scope
\end{verbatim}

These semantics are due to most container iterators being implemented
with \texttt{\&} and \texttt{\&mut}.

\subsection{Iterator adaptors}\label{iterator-adaptors}

The \texttt{Iterator} trait provides many common algorithms as default
methods. For example, the \texttt{fold} method will accumulate the items
yielded by an \texttt{Iterator} into a single value:

\begin{verbatim}
let xs = [1i, 9, 2, 3, 14, 12];
let result = xs.iter().fold(0, |accumulator, item| accumulator - *item);
assert_eq!(result, -41);
\end{verbatim}

Most adaptors return an adaptor object implementing the
\texttt{Iterator} trait itself:

\begin{verbatim}
let xs = [1i, 9, 2, 3, 14, 12];
let ys = [5i, 2, 1, 8];
let sum = xs.iter().chain(ys.iter()).fold(0, |a, b| a + *b);
assert_eq!(sum, 57);
\end{verbatim}

Some iterator adaptors may return \texttt{None} before exhausting the
underlying iterator. Additionally, if these iterator adaptors are called
again after returning \texttt{None}, they may call their underlying
iterator again even if the adaptor will continue to return \texttt{None}
forever. This may not be desired if the underlying iterator has
side-effects.

In order to provide a guarantee about behavior once \texttt{None} has
been returned, an iterator adaptor named \texttt{fuse()} is provided.
This returns an iterator that will never call its underlying iterator
again once \texttt{None} has been returned:

\begin{verbatim}
let xs = [1i,2,3,4,5];
let mut calls = 0i;

{
    let it = xs.iter().scan((), |_, x| {
        calls += 1;
        if *x < 3 { Some(x) } else { None }});

    // the iterator will only yield 1 and 2 before returning None
    // If we were to call it 5 times, calls would end up as 5, despite
    // only 2 values being yielded (and therefore 3 unique calls being
    // made). The fuse() adaptor can fix this.

    let mut it = it.fuse();
    it.next();
    it.next();
    it.next();
    it.next();
    it.next();
}

assert_eq!(calls, 3);
\end{verbatim}

\subsection{For loops}\label{for-loops}

The function \texttt{range} (or \texttt{range\_inclusive}) allows to
simply iterate through a given range:

\begin{verbatim}
for i in range(0i, 5) {
  print!("{} ", i) // prints "0 1 2 3 4"
}

for i in std::iter::range_inclusive(0i, 5) { // needs explicit import
  print!("{} ", i) // prints "0 1 2 3 4 5"
}
\end{verbatim}

The \texttt{for} keyword can be used as sugar for iterating through any
iterator:

\begin{verbatim}
let xs = [2u, 3, 5, 7, 11, 13, 17];

// print out all the elements in the vector
for x in xs.iter() {
    println!("{}", *x)
}

// print out all but the first 3 elements in the vector
for x in xs.iter().skip(3) {
    println!("{}", *x)
}
\end{verbatim}

For loops are \emph{often} used with a temporary iterator object, as
above. They can also advance the state of an iterator in a mutable
location:

\begin{verbatim}
let xs = [1i, 2, 3, 4, 5];
let ys = ["foo", "bar", "baz", "foobar"];

// create an iterator yielding tuples of elements from both vectors
let mut it = xs.iter().zip(ys.iter());

// print out the pairs of elements up to (&3, &"baz")
for (x, y) in it {
    println!("{} {}", *x, *y);

    if *x == 3 {
        break;
    }
}

// yield and print the last pair from the iterator
println!("last: {}", it.next());

// the iterator is now fully consumed
assert!(it.next().is_none());
\end{verbatim}

\subsection{Conversion}\label{conversion}

Iterators offer generic conversion to containers with the
\texttt{collect} adaptor:

\begin{verbatim}
let xs = [0i, 1, 1, 2, 3, 5, 8];
let ys = xs.iter().rev().skip(1).map(|&x| x * 2).collect::<Vec<int>>();
assert_eq!(ys, vec![10, 6, 4, 2, 2, 0]);
\end{verbatim}

The method requires a type hint for the container type, if the
surrounding code does not provide sufficient information.

Containers can provide conversion from iterators through
\texttt{collect} by implementing the \texttt{FromIterator} trait. For
example, the implementation for vectors is as follows:

\begin{verbatim}
impl<T> FromIterator<T> for Vec<T> {
    fn from_iter<I:Iterator<A>>(mut iterator: I) -> Vec<T> {
        let (lower, _) = iterator.size_hint();
        let mut vector = Vec::with_capacity(lower);
        for element in iterator {
            vector.push(element);
        }
        vector
    }
}
\end{verbatim}

\subsubsection{Size hints}\label{size-hints}

The \texttt{Iterator} trait provides a \texttt{size\_hint} default
method, returning a lower bound and optionally on upper bound on the
length of the iterator:

\begin{verbatim}
fn size_hint(&self) -> (uint, Option<uint>) { (0, None) }
\end{verbatim}

The vector implementation of \texttt{FromIterator} from above uses the
lower bound to pre-allocate enough space to hold the minimum number of
elements the iterator will yield.

The default implementation is always correct, but it should be
overridden if the iterator can provide better information.

The \texttt{ZeroStream} from earlier can provide an exact lower and
upper bound:

\begin{verbatim}
# fn main() {}
/// A stream of N zeroes
struct ZeroStream {
    remaining: uint
}

impl ZeroStream {
    fn new(n: uint) -> ZeroStream {
        ZeroStream { remaining: n }
    }

    fn size_hint(&self) -> (uint, Option<uint>) {
        (self.remaining, Some(self.remaining))
    }
}

impl Iterator<int> for ZeroStream {
    fn next(&mut self) -> Option<int> {
        if self.remaining == 0 {
            None
        } else {
            self.remaining -= 1;
            Some(0)
        }
    }
}
\end{verbatim}

\subsection{Double-ended iterators}\label{double-ended-iterators}

The \texttt{DoubleEndedIterator} trait represents an iterator able to
yield elements from either end of a range. It inherits from the
\texttt{Iterator} trait and extends it with the \texttt{next\_back}
function.

A \texttt{DoubleEndedIterator} can have its direction changed with the
\texttt{rev} adaptor, returning another \texttt{DoubleEndedIterator}
with \texttt{next} and \texttt{next\_back} exchanged.

\begin{verbatim}
let xs = [1i, 2, 3, 4, 5, 6];
let mut it = xs.iter();
println!("{}", it.next()); // prints `Some(1)`
println!("{}", it.next()); // prints `Some(2)`
println!("{}", it.next_back()); // prints `Some(6)`

// prints `5`, `4` and `3`
for &x in it.rev() {
    println!("{}", x)
}
\end{verbatim}

The \texttt{chain}, \texttt{map}, \texttt{filter}, \texttt{filter\_map}
and \texttt{inspect} adaptors are \texttt{DoubleEndedIterator}
implementations if the underlying iterators are.

\begin{verbatim}
let xs = [1i, 2, 3, 4];
let ys = [5i, 6, 7, 8];
let mut it = xs.iter().chain(ys.iter()).map(|&x| x * 2);

println!("{}", it.next()); // prints `Some(2)`

// prints `16`, `14`, `12`, `10`, `8`, `6`, `4`
for x in it.rev() {
    println!("{}", x);
}
\end{verbatim}

The \texttt{reverse\_} method is also available for any double-ended
iterator yielding mutable references. It can be used to reverse a
container in-place. Note that the trailing underscore is a workaround
for issue \#5898 and will be removed.

\begin{verbatim}
let mut ys = [1i, 2, 3, 4, 5];
ys.mut_iter().reverse_();
assert!(ys == [5i, 4, 3, 2, 1]);
\end{verbatim}

\subsection{Random-access iterators}\label{random-access-iterators}

The \texttt{RandomAccessIterator} trait represents an iterator offering
random access to the whole range. The \texttt{indexable} method
retrieves the number of elements accessible with the \texttt{idx}
method.

The \texttt{chain} adaptor is an implementation of
\texttt{RandomAccessIterator} if the underlying iterators are.

\begin{verbatim}
let xs = [1i, 2, 3, 4, 5];
let ys = [7i, 9, 11];
let mut it = xs.iter().chain(ys.iter());
println!("{}", it.idx(0)); // prints `Some(1)`
println!("{}", it.idx(5)); // prints `Some(7)`
println!("{}", it.idx(7)); // prints `Some(11)`
println!("{}", it.idx(8)); // prints `None`

// yield two elements from the beginning, and one from the end
it.next();
it.next();
it.next_back();

println!("{}", it.idx(0)); // prints `Some(3)`
println!("{}", it.idx(4)); // prints `Some(9)`
println!("{}", it.idx(6)); // prints `None`
\end{verbatim}

\end{document}
