\documentclass[]{article}
\usepackage{lmodern}
\usepackage{amssymb,amsmath}
\usepackage{ifxetex,ifluatex}
\usepackage{fixltx2e} % provides \textsubscript
\ifnum 0\ifxetex 1\fi\ifluatex 1\fi=0 % if pdftex
  \usepackage[T1]{fontenc}
  \usepackage[utf8]{inputenc}
\else % if luatex or xelatex
  \ifxetex
    \usepackage{mathspec}
    \usepackage{xltxtra,xunicode}
  \else
    \usepackage{fontspec}
  \fi
  \defaultfontfeatures{Mapping=tex-text,Scale=MatchLowercase}
  \newcommand{\euro}{€}
\fi
% use upquote if available, for straight quotes in verbatim environments
\IfFileExists{upquote.sty}{\usepackage{upquote}}{}
% use microtype if available
\IfFileExists{microtype.sty}{\usepackage{microtype}}{}
\usepackage{color}
\usepackage{fancyvrb}
\newcommand{\VerbBar}{|}
\newcommand{\VERB}{\Verb[commandchars=\\\{\}]}
\DefineVerbatimEnvironment{Highlighting}{Verbatim}{commandchars=\\\{\}}
% Add ',fontsize=\small' for more characters per line
\newenvironment{Shaded}{}{}
\newcommand{\KeywordTok}[1]{\textcolor[rgb]{0.00,0.44,0.13}{\textbf{{#1}}}}
\newcommand{\DataTypeTok}[1]{\textcolor[rgb]{0.56,0.13,0.00}{{#1}}}
\newcommand{\DecValTok}[1]{\textcolor[rgb]{0.25,0.63,0.44}{{#1}}}
\newcommand{\BaseNTok}[1]{\textcolor[rgb]{0.25,0.63,0.44}{{#1}}}
\newcommand{\FloatTok}[1]{\textcolor[rgb]{0.25,0.63,0.44}{{#1}}}
\newcommand{\CharTok}[1]{\textcolor[rgb]{0.25,0.44,0.63}{{#1}}}
\newcommand{\StringTok}[1]{\textcolor[rgb]{0.25,0.44,0.63}{{#1}}}
\newcommand{\CommentTok}[1]{\textcolor[rgb]{0.38,0.63,0.69}{\textit{{#1}}}}
\newcommand{\OtherTok}[1]{\textcolor[rgb]{0.00,0.44,0.13}{{#1}}}
\newcommand{\AlertTok}[1]{\textcolor[rgb]{1.00,0.00,0.00}{\textbf{{#1}}}}
\newcommand{\FunctionTok}[1]{\textcolor[rgb]{0.02,0.16,0.49}{{#1}}}
\newcommand{\RegionMarkerTok}[1]{{#1}}
\newcommand{\ErrorTok}[1]{\textcolor[rgb]{1.00,0.00,0.00}{\textbf{{#1}}}}
\newcommand{\NormalTok}[1]{{#1}}
\ifxetex
  \usepackage[setpagesize=false, % page size defined by xetex
              unicode=false, % unicode breaks when used with xetex
              xetex]{hyperref}
\else
  \usepackage[unicode=true]{hyperref}
\fi
\hypersetup{breaklinks=true,
            bookmarks=true,
            pdfauthor={},
            pdftitle={The Rust Pointer Guide},
            colorlinks=true,
            citecolor=blue,
            urlcolor=blue,
            linkcolor=magenta,
            pdfborder={0 0 0}}
\urlstyle{same}  % don't use monospace font for urls
\usepackage[normalem]{ulem}
% avoid problems with \sout in headers with hyperref:
\pdfstringdefDisableCommands{\renewcommand{\sout}{}}
\setlength{\parindent}{0pt}
\setlength{\parskip}{6pt plus 2pt minus 1pt}
\setlength{\emergencystretch}{3em}  % prevent overfull lines
\setcounter{secnumdepth}{5}

\title{The Rust Pointer Guide}

\begin{document}
\maketitle

0.11.0-pre (169c988d09a9d4e46de2b7fead9489e94964c7c7 2014-07-02 18:41:38 +0000)

Copyright © 2011-2014 The Rust Project Developers. Licensed under the
\href{http://www.apache.org/licenses/LICENSE-2.0}{Apache License,
Version 2.0} or the \href{http://opensource.org/licenses/MIT}{MIT
license}, at your option.

This file may not be copied, modified, or distributed except according
to those terms.

{
\hypersetup{linkcolor=black}
\setcounter{tocdepth}{3}
\tableofcontents
}
Rust's pointers are one of its more unique and compelling features.
Pointers are also one of the more confusing topics for newcomers to
Rust. They can also be confusing for people coming from other languages
that support pointers, such as C++. This guide will help you understand
this important topic.

\section{You don't actually need pointers, use
references}\label{you-dont-actually-need-pointers-use-references}

I have good news for you: you probably don't need to care about
pointers, especially as you're getting started. Think of it this way:
Rust is a language that emphasizes safety. Pointers, as the joke goes,
are very pointy: it's easy to accidentally stab yourself. Therefore,
Rust is made in a way such that you don't need them very often.

``But guide!'' you may cry. ``My co-worker wrote a function that looks
like this:

\begin{Shaded}
\begin{Highlighting}[]
\KeywordTok{fn} \NormalTok{succ(x: &}\KeywordTok{int}\NormalTok{) -> }\KeywordTok{int} \NormalTok{\{ *x + }\DecValTok{1} \NormalTok{\}}
\end{Highlighting}
\end{Shaded}

So I wrote this code to try it out:

\textsubscript{\sout{rust\{.ignore\} fn main() \{ let number = 5; let
succ\_number = succ(number); println!(``\{\}'', succ\_number); \}}}

And now I get an error:

\begin{verbatim}
error: mismatched types: expected `&int` but found `<generic integer #0>` (expected &-ptr but found integral variable)
\end{verbatim}

What gives? It needs a pointer! Therefore I have to use pointers!"

Turns out, you don't. All you need is a reference. Try this on for size:

\begin{Shaded}
\begin{Highlighting}[]
\NormalTok{# }\KeywordTok{fn} \NormalTok{succ(x: &}\KeywordTok{int}\NormalTok{) -> }\KeywordTok{int} \NormalTok{\{ *x + }\DecValTok{1} \NormalTok{\}}
\KeywordTok{fn} \NormalTok{main() \{}
    \KeywordTok{let} \NormalTok{number = }\DecValTok{5}\NormalTok{;}
    \KeywordTok{let} \NormalTok{succ_number = succ(&number);}
    \OtherTok{println!}\NormalTok{(}\StringTok{"\{\}"}\NormalTok{, succ_number);}
\NormalTok{\}}
\end{Highlighting}
\end{Shaded}

It's that easy! One extra little \texttt{\&} there. This code will run,
and print \texttt{6}.

That's all you need to know. Your co-worker could have written the
function like this:

\begin{Shaded}
\begin{Highlighting}[]
\KeywordTok{fn} \NormalTok{succ(x: }\KeywordTok{int}\NormalTok{) -> }\KeywordTok{int} \NormalTok{\{ x + }\DecValTok{1} \NormalTok{\}}

\KeywordTok{fn} \NormalTok{main() \{}
    \KeywordTok{let} \NormalTok{number = }\DecValTok{5}\NormalTok{;}
    \KeywordTok{let} \NormalTok{succ_number = succ(number);}
    \OtherTok{println!}\NormalTok{(}\StringTok{"\{\}"}\NormalTok{, succ_number);}
\NormalTok{\}}
\end{Highlighting}
\end{Shaded}

No pointers even needed. Then again, this is a simple example. I assume
that your real-world \texttt{succ} function is more complicated, and
maybe your co-worker had a good reason for \texttt{x} to be a pointer of
some kind. In that case, references are your best friend. Don't worry
about it, life is too short.

However.

Here are the use-cases for pointers. I've prefixed them with the name of
the pointer that satisfies that use-case:

\begin{enumerate}
\def\labelenumi{\arabic{enumi}.}
\item
  Owned: \texttt{Box\textless{}Trait\textgreater{}} must be a pointer,
  because you don't know the size of the object, so indirection is
  mandatory.
\item
  Owned: You need a recursive data structure. These can be infinite
  sized, so indirection is mandatory.
\item
  Owned: A very, very, very rare situation in which you have a
  \emph{huge} chunk of data that you wish to pass to many methods.
  Passing a pointer will make this more efficient. If you're coming from
  another language where this technique is common, such as C++, please
  read ``A note\ldots{}'' below.
\item
  Reference: You're writing a function, and you need a pointer, but you
  don't care about its ownership. If you make the argument a reference,
  callers can send in whatever kind they want.
\item
  Shared: You need to share data among tasks. You can achieve that via
  the \texttt{Rc} and \texttt{Arc} types.
\end{enumerate}

Five exceptions. That's it. Otherwise, you shouldn't need them. Be
sceptical of pointers in Rust: use them for a deliberate purpose, not
just to make the compiler happy.

\subsection{A note for those proficient in
pointers}\label{a-note-for-those-proficient-in-pointers}

If you're coming to Rust from a language like C or C++, you may be used
to passing things by reference, or passing things by pointer. In some
languages, like Java, you can't even have objects without a pointer to
them. Therefore, if you were writing this Rust code:

\begin{Shaded}
\begin{Highlighting}[]
\NormalTok{# }\KeywordTok{fn} \NormalTok{transform(p: Point) -> Point \{ p \}}
\OtherTok{#[}\NormalTok{deriving}\OtherTok{(}\NormalTok{Show}\OtherTok{)]}
\KeywordTok{struct} \NormalTok{Point \{}
    \NormalTok{x: }\KeywordTok{int}\NormalTok{,}
    \NormalTok{y: }\KeywordTok{int}\NormalTok{,}
\NormalTok{\}}

\KeywordTok{fn} \NormalTok{main() \{}
    \KeywordTok{let} \NormalTok{p0 = Point \{ x: }\DecValTok{5}\NormalTok{, y: }\DecValTok{10}\NormalTok{\};}
    \KeywordTok{let} \NormalTok{p1 = transform(p0);}
    \OtherTok{println!}\NormalTok{(}\StringTok{"\{\}"}\NormalTok{, p1);}
\NormalTok{\}}
\end{Highlighting}
\end{Shaded}

I think you'd implement \texttt{transform} like this:

\begin{Shaded}
\begin{Highlighting}[]
\NormalTok{# }\KeywordTok{struct} \NormalTok{Point \{}
\NormalTok{#     x: }\KeywordTok{int}\NormalTok{,}
\NormalTok{#     y: }\KeywordTok{int}\NormalTok{,}
\NormalTok{# \}}
\NormalTok{# }\KeywordTok{let} \NormalTok{p0 = Point \{ x: }\DecValTok{5}\NormalTok{, y: }\DecValTok{10}\NormalTok{\};}
\KeywordTok{fn} \NormalTok{transform(p: &Point) -> Point \{}
    \NormalTok{Point \{ x: p.x + }\DecValTok{1}\NormalTok{, y: p.y + }\DecValTok{1}\NormalTok{\}}
\NormalTok{\}}

\CommentTok{// and change this:}
\KeywordTok{let} \NormalTok{p1 = transform(&p0);}
\end{Highlighting}
\end{Shaded}

This does work, but you don't need to create those references! The
better way to write this is simply:

\begin{Shaded}
\begin{Highlighting}[]
\OtherTok{#[}\NormalTok{deriving}\OtherTok{(}\NormalTok{Show}\OtherTok{)]}
\KeywordTok{struct} \NormalTok{Point \{}
    \NormalTok{x: }\KeywordTok{int}\NormalTok{,}
    \NormalTok{y: }\KeywordTok{int}\NormalTok{,}
\NormalTok{\}}

\KeywordTok{fn} \NormalTok{transform(p: Point) -> Point \{}
    \NormalTok{Point \{ x: p.x + }\DecValTok{1}\NormalTok{, y: p.y + }\DecValTok{1}\NormalTok{\}}
\NormalTok{\}}

\KeywordTok{fn} \NormalTok{main() \{}
    \KeywordTok{let} \NormalTok{p0 = Point \{ x: }\DecValTok{5}\NormalTok{, y: }\DecValTok{10}\NormalTok{\};}
    \KeywordTok{let} \NormalTok{p1 = transform(p0);}
    \OtherTok{println!}\NormalTok{(}\StringTok{"\{\}"}\NormalTok{, p1);}
\NormalTok{\}}
\end{Highlighting}
\end{Shaded}

But won't this be inefficient? Well, that's a complicated question, but
it's important to know that Rust, like C and C++, store aggregate data
types `unboxed,' whereas languages like Java and Ruby store these types
as `boxed.' For smaller structs, this way will be more efficient. For
larger ones, it may be less so. But don't reach for that pointer until
you must! Make sure that the struct is large enough by performing some
tests before you add in the complexity of pointers.

\section{Owned Pointers}\label{owned-pointers}

Owned pointers are the conceptually simplest kind of pointer in Rust. A
rough approximation of owned pointers follows:

\begin{enumerate}
\def\labelenumi{\arabic{enumi}.}
\item
  Only one owned pointer may exist to a particular place in memory. It
  may be borrowed from that owner, however.
\item
  The Rust compiler uses static analysis to determine where the pointer
  is in scope, and handles allocating and de-allocating that memory.
  Owned pointers are not garbage collected.
\end{enumerate}

These two properties make for three use cases.

\subsection{References to Traits}\label{references-to-traits}

Traits must be referenced through a pointer, because the struct that
implements the trait may be a different size than a different struct
that implements the trait. Therefore, unboxed traits don't make any
sense, and aren't allowed.

\subsection{Recursive Data Structures}\label{recursive-data-structures}

Sometimes, you need a recursive data structure. The simplest is known as
a `cons list':

\begin{Shaded}
\begin{Highlighting}[]
\OtherTok{#[}\NormalTok{deriving}\OtherTok{(}\NormalTok{Show}\OtherTok{)]}
\KeywordTok{enum} \NormalTok{List<T> \{}
    \KeywordTok{Nil}\NormalTok{,}
    \KeywordTok{Cons}\NormalTok{(T, Box<List<T>>),}
\NormalTok{\}}

\KeywordTok{fn} \NormalTok{main() \{}
    \KeywordTok{let} \NormalTok{list: List<}\KeywordTok{int}\NormalTok{> = }\KeywordTok{Cons}\NormalTok{(}\DecValTok{1}\NormalTok{, box }\KeywordTok{Cons}\NormalTok{(}\DecValTok{2}\NormalTok{, box }\KeywordTok{Cons}\NormalTok{(}\DecValTok{3}\NormalTok{, box }\KeywordTok{Nil}\NormalTok{)));}
    \OtherTok{println!}\NormalTok{(}\StringTok{"\{\}"}\NormalTok{, list);}
\NormalTok{\}}
\end{Highlighting}
\end{Shaded}

This prints:

\begin{verbatim}
Cons(1, box Cons(2, box Cons(3, box Nil)))
\end{verbatim}

The inner lists \emph{must} be an owned pointer, because we can't know
how many elements are in the list. Without knowing the length, we don't
know the size, and therefore require the indirection that pointers
offer.

\subsection{Efficiency}\label{efficiency}

This should almost never be a concern, but because creating an owned
pointer boxes its value, it therefore makes referring to the value the
size of the box. This may make passing an owned pointer to a function
less expensive than passing the value itself. Don't worry yourself with
this case until you've proved that it's an issue through benchmarks.

For example, this will work:

\begin{Shaded}
\begin{Highlighting}[]
\KeywordTok{struct} \NormalTok{Point \{}
    \NormalTok{x: }\KeywordTok{int}\NormalTok{,}
    \NormalTok{y: }\KeywordTok{int}\NormalTok{,}
\NormalTok{\}}

\KeywordTok{fn} \NormalTok{main() \{}
    \KeywordTok{let} \NormalTok{a = Point \{ x: }\DecValTok{10}\NormalTok{, y: }\DecValTok{20} \NormalTok{\};}
    \NormalTok{spawn(proc() \{}
        \OtherTok{println!}\NormalTok{(}\StringTok{"\{\}"}\NormalTok{, a.x);}
    \NormalTok{\});}
\NormalTok{\}}
\end{Highlighting}
\end{Shaded}

This struct is tiny, so it's fine. If \texttt{Point} were large, this
would be more efficient:

\begin{Shaded}
\begin{Highlighting}[]
\KeywordTok{struct} \NormalTok{Point \{}
    \NormalTok{x: }\KeywordTok{int}\NormalTok{,}
    \NormalTok{y: }\KeywordTok{int}\NormalTok{,}
\NormalTok{\}}

\KeywordTok{fn} \NormalTok{main() \{}
    \KeywordTok{let} \NormalTok{a = box Point \{ x: }\DecValTok{10}\NormalTok{, y: }\DecValTok{20} \NormalTok{\};}
    \NormalTok{spawn(proc() \{}
        \OtherTok{println!}\NormalTok{(}\StringTok{"\{\}"}\NormalTok{, a.x);}
    \NormalTok{\});}
\NormalTok{\}}
\end{Highlighting}
\end{Shaded}

Now it'll be copying a pointer-sized chunk of memory rather than the
whole struct.

\section{References}\label{references}

References are the third major kind of pointer Rust supports. They are
simultaneously the simplest and the most complicated kind. Let me
explain: references are considered `borrowed' because they claim no
ownership over the data they're pointing to. They're just borrowing it
for a while. So in that sense, they're simple: just keep whatever
ownership the data already has. For example:

\begin{Shaded}
\begin{Highlighting}[]
\KeywordTok{struct} \NormalTok{Point \{}
    \NormalTok{x: }\KeywordTok{f32}\NormalTok{,}
    \NormalTok{y: }\KeywordTok{f32}\NormalTok{,}
\NormalTok{\}}

\KeywordTok{fn} \NormalTok{compute_distance(p1: &Point, p2: &Point) -> }\KeywordTok{f32} \NormalTok{\{}
    \KeywordTok{let} \NormalTok{x_d = p1.x - p2.x;}
    \KeywordTok{let} \NormalTok{y_d = p1.y - p2.y;}

    \NormalTok{(x_d * x_d + y_d * y_d).sqrt()}
\NormalTok{\}}

\KeywordTok{fn} \NormalTok{main() \{}
    \KeywordTok{let} \NormalTok{origin =    &Point \{ x: }\DecValTok{0.0}\NormalTok{, y: }\DecValTok{0.0} \NormalTok{\};}
    \KeywordTok{let} \NormalTok{p1     = box Point \{ x: }\DecValTok{5.0}\NormalTok{, y: }\DecValTok{3.0} \NormalTok{\};}

    \OtherTok{println!}\NormalTok{(}\StringTok{"\{\}"}\NormalTok{, compute_distance(origin, p1));}
\NormalTok{\}}
\end{Highlighting}
\end{Shaded}

This prints \texttt{5.83095189}. You can see that the
\texttt{compute\_distance} function takes in two references, a reference
to a value on the stack, and a reference to a value in a box. Of course,
if this were a real program, we wouldn't have any of these pointers,
they're just there to demonstrate the concepts.

So how is this hard? Well, because we're ignoring ownership, the
compiler needs to take great care to make sure that everything is safe.
Despite their complete safety, a reference's representation at runtime
is the same as that of an ordinary pointer in a C program. They
introduce zero overhead. The compiler does all safety checks at compile
time.

This theory is called `region pointers' and you can read more about it
\href{http://www.cs.umd.edu/projects/cyclone/papers/cyclone-regions.pdf}{here}.
Region pointers evolved into what we know today as `lifetimes'.

Here's the simple explanation: would you expect this code to compile?

\textsubscript{\sout{rust\{.ignore\} fn main() \{ println!(``\{\}'', x);
let x = 5; \}}}

Probably not. That's because you know that the name \texttt{x} is valid
from where it's declared to when it goes out of scope. In this case,
that's the end of the \texttt{main} function. So you know this code will
cause an error. We call this duration a `lifetime'. Let's try a more
complex example:

\begin{Shaded}
\begin{Highlighting}[]
\KeywordTok{fn} \NormalTok{main() \{}
    \KeywordTok{let} \KeywordTok{mut} \NormalTok{x = box }\DecValTok{5i}\NormalTok{;}
    \KeywordTok{if} \NormalTok{*x < }\DecValTok{10} \NormalTok{\{}
        \KeywordTok{let} \NormalTok{y = &x;}
        \OtherTok{println!}\NormalTok{(}\StringTok{"Oh no: \{\}"}\NormalTok{, y);}
        \KeywordTok{return}\NormalTok{;}
    \NormalTok{\}}
    \NormalTok{*x -= }\DecValTok{1}\NormalTok{;}
    \OtherTok{println!}\NormalTok{(}\StringTok{"Oh no: \{\}"}\NormalTok{, x);}
\NormalTok{\}}
\end{Highlighting}
\end{Shaded}

Here, we're borrowing a pointer to \texttt{x} inside of the \texttt{if}.
The compiler, however, is able to determine that that pointer will go
out of scope without \texttt{x} being mutated, and therefore, lets us
pass. This wouldn't work:

\textsubscript{\textasciitilde{}}rust\{.ignore\} fn main() \{ let mut x
= box 5i; if \emph{x \textless{} 10 \{ let y = \&x; }x -= 1;

\begin{verbatim}
    println!("Oh no: {}", y);
    return;
}
*x -= 1;
println!("Oh no: {}", x);
\end{verbatim}

\} \textsubscript{\textasciitilde{}}

It gives this error:

\begin{verbatim}
test.rs:5:8: 5:10 error: cannot assign to `*x` because it is borrowed
test.rs:5         *x -= 1;
                  ^~
test.rs:4:16: 4:18 note: borrow of `*x` occurs here
test.rs:4         let y = &x;
                          ^~
\end{verbatim}

As you might guess, this kind of analysis is complex for a human, and
therefore hard for a computer, too! There is an entire
\href{guide-lifetimes.html}{guide devoted to references and lifetimes}
that goes into lifetimes in great detail, so if you want the full
details, check that out.

\section{Returning Pointers}\label{returning-pointers}

We've talked a lot about functions that accept various kinds of
pointers, but what about returning them? In general, it is better to let
the caller decide how to use a function's output, instead of assuming a
certain type of pointer is best.

What does that mean? Don't do this:

\begin{Shaded}
\begin{Highlighting}[]
\KeywordTok{fn} \NormalTok{foo(x: Box<}\KeywordTok{int}\NormalTok{>) -> Box<}\KeywordTok{int}\NormalTok{> \{}
    \KeywordTok{return} \NormalTok{box *x;}
\NormalTok{\}}

\KeywordTok{fn} \NormalTok{main() \{}
    \KeywordTok{let} \NormalTok{x = box }\DecValTok{5}\NormalTok{;}
    \KeywordTok{let} \NormalTok{y = foo(x);}
\NormalTok{\}}
\end{Highlighting}
\end{Shaded}

Do this:

\begin{Shaded}
\begin{Highlighting}[]
\KeywordTok{fn} \NormalTok{foo(x: Box<}\KeywordTok{int}\NormalTok{>) -> }\KeywordTok{int} \NormalTok{\{}
    \KeywordTok{return} \NormalTok{*x;}
\NormalTok{\}}

\KeywordTok{fn} \NormalTok{main() \{}
    \KeywordTok{let} \NormalTok{x = box }\DecValTok{5}\NormalTok{;}
    \KeywordTok{let} \NormalTok{y = box foo(x);}
\NormalTok{\}}
\end{Highlighting}
\end{Shaded}

This gives you flexibility, without sacrificing performance.

You may think that this gives us terrible performance: return a value
and then immediately box it up ?! Isn't that the worst of both worlds?
Rust is smarter than that. There is no copy in this code. \texttt{main}
allocates enough room for the \texttt{box int}, passes a pointer to that
memory into \texttt{foo} as \texttt{x}, and then \texttt{foo} writes the
value straight into that pointer. This writes the return value directly
into the allocated box.

This is important enough that it bears repeating: pointers are not for
optimizing returning values from your code. Allow the caller to choose
how they want to use your output.

\section{Related Resources}\label{related-resources}

\begin{itemize}
\itemsep1pt\parskip0pt\parsep0pt
\item
  \href{guide-lifetimes.html}{Lifetimes guide}
\end{itemize}

\end{document}
